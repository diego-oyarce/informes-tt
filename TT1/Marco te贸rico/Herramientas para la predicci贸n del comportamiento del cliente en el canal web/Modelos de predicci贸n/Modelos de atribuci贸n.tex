\subsubsection{Modelos de atribución}

Los modelos de atribución permiten predecir el recorrido que los clientes seguirán al momento de concretar una compra. Este recorrido puede contener las redes sociales, el uso del sitio web del vendedor, el correo electrónico, entre otros. Los modelos de atribución permiten determinar el impacto que tiene el uso de las acciones para el sistema de marketing. Este tipo de modelo permite darle mayor importancia a los canales de marketing y a los puntos de contacto que existen entre el cliente y el vendedor, que llevaron al cliente a realizar una compra.

Al asignar crédito a sus canales de marketing y puntos de contacto, se puede aumentar la posibilidad de que los clientes logren concretar una compra, esto a través de la identificación de las áreas del recorrido del comprador que se puedan mejorar, la determinación del retorno de la inversión para cada canal o punto de contacto, el descubrimiento de las áreas más efectivas para gastar el presupuesto de marketing y la adaptación de las campañas de marketing y muestra de contenido totalmente personalizado por clientes \cite{modelo-atribución}.

Existen variados tipos de modelos de atribución, todos tienen el mismo procedimiento de asignar crédito a los canales y punto de contacto, cada uno de estos tipos de modelo le atribuyen un peso distinto a cada canal y punto de contacto. Los modelos a continuación son los más aptos para lograr la predicción del comportamiento de un cliente:

\begin{itemize}
    \item Modelo de atribución Multi-Touch: Este modelo demuestra ser poderoso ya que tiene en cuenta todos los canales y puntos de contacto con lo que los clientes interactúan a lo largo de su camino al concretar una compra. Deja en evidencia cuáles de los canales y punto de contacto fueron más influyentes y de cómo estas trabajaron en conjunto para influenciar al cliente.
    \item Modelo de atribución Lineal: Corresponde a un tipo de modelo de atribución Multi-Touch que le entrega el mismo peso a cada uno de los canales y puntos de contacto con los que el cliente interactúa en su camino al concretar una compra.
    \item Modelo de atribución Time-Decay: También llamado modelo de atribución de declive en el tiempo, además de considerar todos los puntos de contacto, también considera el tiempo que cada uno de estos puntos de contacto ocurrió, por lo que, los puntos de contacto o interacciones que sucedieron más cercano al momento en que se concretó la compra reciben mayor peso.
\end{itemize}

\subsubsection{Ventajas de los modelos de atribución}

\begin{itemize}
    \item Facilita el rastrear de mejor manera el paso a paso del cliente: Esto gracias a  la atención que se le entrega a cada canal y punto de contacto con el cual el cliente interactúa a la hora de concretar una compra.
    \item Permiten mayor personalización de rastreo de los clientes: Al saber que canales y punto de contacto tiene cada uno de los clientes, se puede llegar a entregar una experiencia personalizada a cada uno de los clientes.
    \item Comprender la contribución de cada canal y punto de contacto: Permite comprender como cada canal y punto de contacto contribuye a lograr los objetivos comerciales. Siendo de gran ayuda para identificar como asignar los recursos de manera mas efectiva y lograr optimizar las estrategias.
    \item Identificar canales y puntos de contacto de alto rendimiento: Un modelo de atribución puede revelar qué canales o puntos de contacto tienen un mayor interacción con los clientes en términos de generación de resultados. Esto permite a las empresas enfocar sus recursos en los canales más efectivos y maximizar su retorno de inversión.
\end{itemize}

\subsubsection{Desventajas de los modelos de atribución}

\begin{itemize}
    \item Poseen una mayor complejidad que los otros modelos: La implementación de un modelo de atribución puede ser compleja y requerir un enfoque personalizado según las necesidades y características de cada empresa. Además, no hay un modelo de atribución único que sea universalmente aceptado, lo que puede generar falta de consenso y confusión en la industria.
    \item La interpretación de los resultados puede ser subjetiva: La interpretación de los resultados de un modelo de atribución puede estar sujeta a la interpretación y suposiciones del analista. Diferentes personas pueden llegar a conclusiones diferentes basadas en los mismos resultados, lo que puede generar cierta subjetividad en la interpretación de los datos.
    \item Poseen limitaciones en la medición del seguimiento: El modelo de atribución depende de la disponibilidad y calidad de los datos. Si los datos son limitados o imprecisos, los resultados del modelo pueden no ser confiables o representativos de la realidad.
\end{itemize}