El presente documento de propuesta de Trabajo de Titulación tiene como objetivo mostrar la forma y el plan de trabajo que se utilizan a lo largo del proceso de desarrollo del proyecto propuesto.

El proyecto tiene como objetivo fundamental analizar el comportamiento de los clientes de AFP Capital y sus preferencias de uso en un período igual o inferior a 6 meses, para predecir navegaciones futuras personalizadas.

Este proyecto consta de cuatro fases para su desarrollo, las cuales abarcan la planificación y planteamiento de los antecedentes generales para la realización del proyecto, la investigación de la problemática en estudio en base a la situación actual planteada, el modelamiento y desarrollo del proyecto, que abarca el modelamiento de datos y cómo será afrontado el proceso ETL, hasta el desarrollo del código que soportará y hará funcionar el modelo predictivo, en base a la construcción de bases de datos, APIs y realización de pruebas para mitigar los posibles errores encontrados, y la última fase que dará fin al desarrollo del proyecto, es la fase de las conclusiones y recomendaciones, en la cual se darán a conocer las conclusiones que se fueron recabando a lo largo del desarrollo y elaborando un manual de usuario con las recomendaciones de uso.

Además, este proyecto estará bajo un marco de trabajo de desarrollo ágil, Scrum y metodologías de análisis y minería de datos, las cuales son CRISP-DM y OSEMN. El entorno de desarrollo estará basado en Python, junto a los librerías de análisis y minería de datos (Pandas, Numpy, etc.) y frameworks de desarrollo de APIs (Flask, Django y FastAPI).

El proyecto tiene una duración de dos semestres académicos, los cuales abarcan las asignaturas Título I y Título II, en donde se elaborarán como entregables un Informe Final de Trabajo de Título y el sistema (MVP) del proyecto propuesto.