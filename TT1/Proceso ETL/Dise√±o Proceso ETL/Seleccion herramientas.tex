En esta etapa, se realiza la selección de herramientas de software que se ajusten a las necesidades y requisitos del proyecto para llevar a cabo el proceso ETL de manera eficiente. Se evalúan diferentes opciones disponibles en función de su capacidad, compatibilidad y facilidad de uso, para garantizar una elección adecuada, es por esto que se seleccionaron las siguientes herramientas:
\begin{itemize}
    \item \textbf{Visual Studio Code:} Es un editor de código fuente desarrollado por Microsoft. Es conocido por su enfoque en la simplicidad, la personalización y la eficiencia. 
    \item \textbf{Python:} Es un lenguaje de programación interpretado, de alto nivel y de propósito general, conocido por su sintaxis clara y legible. Es utilizado en una amplia gama de aplicaciones, desde desarrollo web hasta ciencia de datos y aprendizaje automático. Python se destaca por su facilidad de aprendizaje y su amplia biblioteca estándar, que ofrece numerosas funcionalidades predefinidas para diversas tareas \cite{python}.
    \begin{itemize}
        \item \textbf{Numpy:} Es una biblioteca fundamental para la computación científica en Python. Proporciona estructuras de datos eficientes y funciones para realizar operaciones numéricas y de manipulación de arrays \cite{numpy}.
        \item \textbf{Pandas:} Es una biblioteca poderosa para el análisis de datos basada en Numpy, que proporciona estructuras de datos flexibles y eficientes, como DataFrames, y un conjunto completo de funciones para la manipulación y transformación de datos \cite{pandas}.
        \item \textbf{Dask:} Es una biblioteca de paralelización flexible que permite escalar el procesamiento de datos en Python. Proporciona estructuras de datos paralelas y operaciones distribuidas que facilitan el procesamiento de grandes volúmenes de datos \cite{dask}.
    \end{itemize}
\end{itemize}
