Dentro de esta etapa se realiza la planificación detallada de las transformaciones necesarias para construir una base sólida y consistente para el desarrollo del proyecto. Estas transformaciones implican una serie de pasos que permiten limpiar, filtrar, combinar y enriquecer los datos de manera adecuada.

La planificación de las transformaciones es fundamental para garantizar la calidad y la integridad de los datos que serán utilizados en el proyecto. Durante esta etapa, se identifican las tareas específicas que deben llevarse a cabo para lograr los objetivos establecidos, teniendo en cuenta los requisitos del proyecto y las necesidades del negocio.

Algunas de las transformaciones comunes incluyen:

\begin{itemize}
    \item \textbf{Limpieza de datos:} Se realizan tareas de limpieza para corregir errores, eliminar valores duplicados o inconsistentes, y garantizar la coherencia de los datos. Esto puede incluir la corrección de formatos incorrectos, la normalización de datos, el manejo de valores faltantes o la estandarización de la información. La limpieza se realizo para solo tomar el cuenta los valores no nulos y se establecío el formato 'YYYY;MM;DD;HH;MM;SS.ss' de la columna 'fecha evento'.

    \item \textbf{Filtrado de datos:} Se aplican filtros para seleccionar y extraer los datos relevantes para el proyecto, descartando aquellos que no cumplen con ciertos criterios o condiciones específicas, \textbf{no permitiendo registros sin métodos asociados}. Esto ayuda a reducir el volumen de datos y a enfocarse en la información más relevante y útil.

    \item \textbf{Combinación de datos:} Se integran datos provenientes de diferentes fuentes o fuentes de datos diversas. Esto implica fusionar conjuntos de datos relacionados, realizar uniones o cruces de tablas, y establecer relaciones entre los datos para generar una visión global y coherente.

    \item \textbf{Enriquecimiento de datos:} Se agregan atributos o información adicional a los datos existentes para enriquecer su contexto y mejorar su valor. Esto puede implicar la incorporación de datos externos, la realización de cálculos derivados, la normalización de datos o la aplicación de reglas específicas.
\end{itemize}


Es importante tener en cuenta que la planificación de las transformaciones considera el orden y la secuencia adecuada de ejecución, así como la documentación de cada paso y los criterios de validación y verificación para garantizar la calidad de los datos transformados.
% Por esto es que se diseño el siguiente plan detallado con las transformaciones a aplicar.
