En esta etapa, se lleva a cabo la implementación del diseño del proceso ETL previamente definido, utilizando las herramientas seleccionadas. Se desarrollan los flujos de extracción, transformación y carga de los datos según lo establecido en el diseño.

Una vez implementado, se procede a realizar pruebas exhaustivas para garantizar el correcto funcionamiento del proceso. Estas pruebas incluyen la verificación de la extracción de datos de las fuentes, la correcta aplicación de las transformaciones definidas y la carga exitosa de los datos en el destino final.

El objetivo de las pruebas es asegurar que el proceso ETL cumpla con los requisitos establecidos y que los resultados obtenidos sean los esperados. Esto implica validar la integridad y coherencia de los datos transformados, así como verificar el rendimiento y la escalabilidad del proceso.

En caso de encontrar inconvenientes o desviaciones durante las pruebas, se realizan los ajustes necesarios en el diseño o en la configuración de las herramientas utilizadas. Es fundamental realizar iteraciones y pruebas adicionales hasta obtener resultados consistentes y satisfactorios.

Como primer paso en la construcción del proceso ETL, se importan las bibliotecas pandas, numpy y date proveniente de la biblioteca integrada datetime. Luego se establece la mes para ser usado como parametro de fecha y se realiza una lectura de los datos creando un dataframe de pandas para poder ser visualizados y procesados.

Como segundo paso se establecío el formato de fecha 'YYYY;MM;DD;HH;MM;SS.ss' para la columna 'fecha evento' y se eliminaron del dataframe todas las filas que contenian valores nulos.


\begin{lstlisting}[language=Python]
import pandas as pd
from datetime import date

#Parametros 
today = date.today()
mes = str(today.strftime("%Y%m"))

df_weblogs=pd.read_csv(r'ruta\archivoacceso.csv',sep=';',encoding='utf-8')

df_weblogs['fecha_evento']=pd.to_datetime(df_weblogs['fecha_evento'],
                                    format="%b %d, %Y @ %H:%M:%S.%f")

df_weblogs=df_weblogs.dropna()

df_weblogs.to_csv(r'ruta\archivodestino.csv'.csv', sep=';', index=False)
\end{lstlisting}

Para finalizar el proceso ETL se guarda el dataframe procesado en un archivo .csv dentro de los archivos del proyecto.