\begin{table}[ht]
\captionsetup{font=small} % Ajusta el tamaño de fuente de la leyenda de la tabla
\small % Ajusta el tamaño de fuente de la tabla
\begin{tabular}{|p{0.2\linewidth}|p{0.27\linewidth}|p{0.27\linewidth}|p{0.26\linewidth}|}
\hline
\textbf{Modelo} & \textbf{Ventajas} & \textbf{Desventajas} & \textbf{Aplicaciones} \\
\hline
Modelos de Regresión & 
Proporciona una relación cuantitativa entre variables independientes y la variable de respuesta. & 
Supone una relación lineal entre variables, lo que puede no ser válido en todos los casos. & 
Predicción de valores numéricos continuos. \\
\hline
Modelos de Recomendación & 
Personalización de sugerencias para los usuarios. & 
Puede requerir una gran cantidad de datos y tener problemas con datos faltantes o sesgos inherentes. & 
Recomendaciones de productos en comercio electrónico. \\
\hline
Modelos de Series Temporales & 
Captura patrones temporales y estacionales en los datos a lo largo del tiempo. & 
Sensibilidad a valores atípicos y datos faltantes, y dificultad para capturar tendencias no lineales. & 
Predicción de la demanda de productos. Pronóstico del clima. \\
\hline
Modelos de Atribución & 
Permite cuantificar la contribución relativa de diferentes variables a un resultado o impacto. & 
Puede ser difícil determinar la verdadera relación causal entre las variables. & 
Evaluación del retorno de inversión (ROI) de una campaña publicitaria. \\
\hline
Modelos de Árboles de Decisión & 
Proporciona una estructura de decisiones fácilmente interpretable. & 
Pueden ser propensos al sobreajuste si no se controla adecuadamente. & 
Clasificación y predicción en diversos campos, como medicina, marketing y finanzas. \\
\hline
\end{tabular}
\end{table}
