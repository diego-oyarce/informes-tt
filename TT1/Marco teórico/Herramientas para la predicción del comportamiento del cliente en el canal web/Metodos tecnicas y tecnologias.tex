%Métodos, técnicas y tecnologías de análisis de datos utilizados en la predicción del comportamiento del cliente
En la actualidad, el análisis de datos desempeña un papel fundamental en la predicción del comportamiento del cliente. Las empresas y organizaciones buscan comprender y anticiparse a las necesidades y preferencias de sus clientes para mejorar la toma de decisiones y ofrecer productos y servicios más personalizados. Para lograr esto, se han desarrollado diversos métodos, técnicas y tecnologías que permiten analizar grandes volúmenes de datos y extraer información valiosa. 
A continuación, se listan algunos de los métodos, técnicas y tecnologías más utilizados en el análisis de datos para predecir el comportamiento del cliente.
\vspace{0.5cm}

\textbf{\large Métodos y modelos}
\begin{itemize}
    \item Regresión logística
    \item Clustering
    \item Árboles de decisión
    \item Random Forest
    \item Gradient Boosting Machine
\end{itemize}
\vspace{0.5cm}

\textbf{\large Técnicas}
\begin{itemize}
    \item Redes neuronales artificiales (ANN)
    \item Support Vector Machine (SVM)
\end{itemize}
\vspace{0.5cm}

\textbf{\large Tecnologías} 
\begin{itemize}
    \item Tableau
    \item Python (con bibliotecas como Pandas, NumPy, Scikit-learn)
    \item R (con paquetes como dplyr, caret, randomForest)
    \item Apache Spark
    \item KNIME
    \item RapidMiner
    \item QlikView
    \item Power BI
\end{itemize}
