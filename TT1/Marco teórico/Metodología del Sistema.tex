\subsubsection{CRISP-DM}
La metodología CRISP-DM (Cross-Industry Standard Process for Data Mining) es un proceso estándar utilizado para realizar proyectos de minería de datos. La metodología CRISP-DM se divide en seis fases distintas que se describen a continuación:

\begin{enumerate}
    \item \textbf{Comprensión del problema:} En esta fase se define el problema a resolver y se establecen los objetivos del proyecto. También se recopilan los datos necesarios para el proyecto.
    \item \textbf{Comprensión de los datos:} En esta fase se realiza una exploración de los datos para comprender su calidad, estructura y relevancia para el problema en cuestión.
    \item \textbf{Preparación de los datos:} En esta fase se limpian y procesan los datos para que puedan ser utilizados en la etapa de modelado.
    \item \textbf{Modelado:} En esta fase se aplican técnicas de modelado para desarrollar un modelo predictivo. Se prueban diferentes modelos y se selecciona el que mejor se ajuste a los datos.
    \item \textbf{Evaluación:} En esta fase se evalúa el modelo desarrollado en la fase anterior. Se verifica que el modelo funcione correctamente y se ajuste adecuadamente a los datos.
    \item \textbf{Implementación:} En esta fase se implementa el modelo desarrollado en la fase de modelado en un entorno de producción. También se establecen planes para monitorear el rendimiento del modelo y actualizarlo según sea necesario.
\end{enumerate}

Las fases de la metodología CRISP-DM son iterativas, lo que significa que es posible volver a una fase anterior si es necesario.

\subsubsection{OSEMN}

La metodología OSEMN (acrónimo de las palabras en inglés: Obtain, Scrub, Explore, Model, Interpret) es un proceso utilizado en la minería de datos y el análisis de datos para trabajar con grandes conjuntos de datos de manera efectiva. 

\begin{enumerate}
    \item \textbf{Obtener (Obtain):} En esta etapa, se recopilan los datos necesarios para el análisis. Los datos pueden provenir de diferentes fuentes, como bases de datos, archivos en línea o registros de sensores. La calidad y la cantidad de los datos obtenidos son cruciales para el éxito del análisis.
    \item \textbf{Limpieza (Scrub):} Una vez que se han obtenido los datos, es necesario realizar una limpieza para eliminar datos innecesarios o incorrectos. Esta etapa puede implicar la eliminación de duplicados, la corrección de errores y la eliminación de valores atípicos. El objetivo de esta etapa es obtener datos limpios y coherentes para el análisis.
    \item \textbf{Exploración (Explore):} En esta etapa, se utilizan técnicas de visualización y estadísticas para explorar los datos y obtener información sobre ellos. Se pueden identificar patrones, tendencias y relaciones entre diferentes variables. El objetivo es obtener una comprensión más profunda de los datos y de cómo se relacionan entre sí.
    \item \textbf{Modelado (Model):} En esta etapa, se utilizan técnicas de modelado estadístico o de aprendizaje automático para crear modelos que puedan predecir resultados futuros o identificar patrones en los datos. El objetivo es utilizar los datos para crear un modelo que pueda utilizarse para tomar decisiones informadas.
    \item \textbf{Interpretación (Interpret):} En esta etapa, se interpretan los resultados obtenidos en la etapa de modelado. Los resultados pueden ser utilizados para tomar decisiones o para generar nuevas hipótesis que puedan ser exploradas en futuros análisis.
\end{enumerate}

Se propone el uso de la metodología OSEMN, ya que se enfoca en el análisis de datos y la creación de modelos predictivos. OSEMN también es una metodología más flexible que CRISP-DM, lo que puede ser útil en un proyecto de SCRUM donde se busca una mayor adaptabilidad.

Por otro lado, también se propone el uso de la metodología CRISP-DM, ya que el proyecto incluye una etapa de exploración y análisis de datos, seguida por una fase de construcción de modelos. CRISP-DM se enfoca en el proceso completo de minería de datos, desde la comprensión del problema hasta la implementación del modelo, lo que puede servir para realizar un trabajo más estructurado.

Ya que este proyecto se encuentra bajo el marco de trabajo SCRUM, ambas metodologías pueden ser utilizadas de manera complementaria, utilizando OSEMN para las fases de creación de modelos y CRISP-DM para la etapa de exploración y análisis de datos.
