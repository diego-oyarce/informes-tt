Para comprender la experiencia y el comportamiento del cliente dentro de un canal web, es importante reconocer la existencia del consumer journey, el cual describe las distintas etapas por las que un cliente pasa al momento de consumo de un producto o servicio. Según Lemon y Verhoef (2016) las etapas corresponden a conciencia, investigación, consideración, compra, uso y evaluación. La definición de conciencia da cuenta de la necesidad o el problema que debe ser resuelto, mientras que investigación refiere de la búsqueda de información por parte del cliente para posibles soluciones, comparando entre las distintas opciones disponibles (Lemon y Verhoef, 2016). Luego la etapa de consideración donde el cliente puede evaluar entre las opciones disponibles escogiendo la que mejor se adapta a sus necesidades dando paso a la etapa de compra cuando el cliente contrata y/o compra el mejor servicio a su parecer. Posterior viene la etapa de uso donde el cliente puede experimentar y testear la calidad, funcionalidad y experiencia del servicio dando pie a la última etapa que consiste en evaluar la experiencia como satisfactoria o insatisfactoria con la entrega voluntaria de feedback tanto positivo como negativo. Por lo tanto las posibles opciones disponibles para los clientes dentro del canal web buscan hacer del consumer journey una eficiente y grata experiencia. 
Para poder acceder al canal web de AFP Capital, se debe estar afiliado y tener una cuenta privada personal [Rut y Contraseña] y una vez se hace ingreso al canal web privado, el afiliado tiene disponibles variadas opciones para realizar y que buscan satisfacer sus posibles necesidades, estas corresponden al pago o no de la cotización mensual, la obtención de certificados de cotizaciones, certificado de afiliación, certificado de antecedentes previsionales, certificados de traspaso de fondos, certificado de vacaciones progresivas y certificados tributarios, como también la obtención de certificados generales, como el certificado de residencia, certificado de suscripción de ahorro previsional voluntario [APV], certificado de cuenta 2, certificado de remuneraciones imponibles, certificado de periodos no cotizados y certificado de trabajo pesado, si el afiliado es una persona pensionada puede obtener certificado de asignacion familiar, certificado de calidad pensionado, certificado de pensiones pagadas, certificado de pensión en trámite, certificado de ingreso base y certificado de comprobante de pago de pensión, también poder hacer obtención de la cartola en línea. El canal web privado permite realizar el ahorro obligatorio y ahorrar voluntariamente, dentro de una cuenta de ahorro previsional voluntario [APV] o cuenta 2, realizar inversiones, hacer depósitos directos, tener planillas de pagos y ver las comisión cobrada como afiliado. También le otorga al afiliado la opción de ver su fondo de pensiones, ver los tipos de fondo de pensión, tipo A, tipo B, tipo C, tipo D, tipo E y sus porcentajes de rentabilidad, realizar un cambio de fondo de pensiones y recibir educación previsional. Le otorga al afiliado la opción de realizar giros en sus cuentas personales, acceder a rescates financieros y realizar el trámite de pensión. 

