Considerando los modelos de negocio establecidos por las Administradoras de Fondos de Pensiones (AFP), surge la importancia de la figura del cliente. Según la Real Academia Española, un cliente es una persona que realiza una compra o utiliza los servicios ofrecidos por un profesional o empresa (Real Academia Española, s.f). Sin embargo, en el contexto de las AFP, los clientes se denominan afiliados, ya que contribuyen o están inscritos en un plan de pensiones \cite{pension-system}.

El afiliado es el centro del negocio y su importancia radica principalmente en la rentabilidad que aporta. Cada trabajador que decide afiliarse representa una ganancia, mientras que cada afiliado que decide desafiliarse genera una pérdida. Además, la experiencia del servicio que brinda la AFP hacia el afiliado es crucial, ya que puede promover la marca si es positiva. En tercer lugar, el afiliado, al ser una fuente de ganancias para el modelo, puede contribuir al crecimiento de la empresa al darle su preferencia. Además, la experiencia del cliente y su retroalimentación son valiosas, ya que pueden proporcionar conocimientos sobre los puntos débiles y las áreas de mejora del sistema \cite{def-cliente}.

Dentro de las diferentes funciones que tiene el cliente, en primer lugar, se encuentra el cliente como consumidor. Esta es una de las funcionalidades más tradicionales, ya que el objetivo intrínseco del cliente es consumir o contratar servicios. Como consumidor, adquiere un producto o servicio y lo utiliza para satisfacer una necesidad, lo que representa la principal fuente de ingresos para la empresa.

En segundo lugar, se encuentra el cliente como "prosumidor", es decir, alguien que consume y produce al mismo tiempo \cite{understand-roles}. Además de consumir, el cliente también deja reseñas o realiza comentarios en lugares especializados, lo cual es útil para generar información que mejore la experiencia del servicio.

En tercer lugar, se considera al cliente como crítico, ya que si la experiencia del cliente es negativa, los comentarios y reseñas negativas que proporcione pueden tener un impacto constructivo o destructivo.

En cuarto lugar, el cliente es una pieza fundamental en el desarrollo de productos y servicios. Los comentarios de los clientes pueden guiar el desarrollo de servicios innovadores que se ajusten a las necesidades que ellos indican. En el caso específico de las AFP, esto se refiere a los afiliados.

En quinto lugar, el cliente se desempeña como evaluador de la experiencia. Relacionado con los puntos anteriores, la mejor manera de mejorar la experiencia del cliente es tener en cuenta sus comentarios sobre este aspecto, lo que puede marcar la diferencia frente a otras empresas competidoras en el mercado.

Por último, el cliente puede convertirse en un embajador eventual de la marca, es decir, puede promover el negocio mediante recomendaciones, comentarios y reseñas positivas.