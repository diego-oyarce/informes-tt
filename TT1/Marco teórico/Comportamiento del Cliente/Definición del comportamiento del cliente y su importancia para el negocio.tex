Considerando los modelos de negocios establecidos por las Administradoras de Fondos de Pensiones [AFP], de ahí radica la importancia de la figura del cliente. Según lo que indica la Real Academia Española, el cliente es la persona que realiza una compra o utiliza los servicios que un profesional o empresa pueda ofrecer (Real Academia Española, s.f), no obstante en base al sistema establecido por las Administradoras de Fondos de Pensiones, el cliente obtiene el nombre de afiliado pues estos contribuyen o se encuentran inscritos en un plan de pensiones (Rasekhi, Fard y Kim, 2016). 
El afiliado es el centro del negocio, cuya gran importancia radica principalmente en la rentabilidad que brinda. Cada trabajador que decida afiliarse se traduce en una ganancia, mientras que cada afiliado que decida desafiliarse genera perdida. Considerando esto es que se puede apreciar la segunda importancia del afiliado, debido a que este promueve la marca si es que la experiencia del servicio de cara al usuario es buena. En tercer lugar, el afiliado, al ser un ganancia para el modelo, este a su vez que obtiene el servicio es capaz de posibilitar el crecimiento de la empresa al tener su preferencia. Por otro lado, la experiencia del cliente y su feedback es valiosa ya que puede brindar conocimiento de los puntos débiles y con posibilidad de mejora que tiene el sistema (Rodriguez, 2023). 
Dentro de las distintas funciones que el cliente tiene, en primer lugar se puede mencionar al cliente como consumidor. Consiste en unas de las funcionalidades más tradicionales puesto que el objetivo intrínseco del cliente es consumir o contratar servicios. Como consumidor es quien adquiere un producto o servicio y lo aprovecha para un fin o necesidad, por lo que la empresa obtiene su principal fuente de ingresos.
En segundo lugar, se tiene al cliente como prosumidor, en otras palabras, consume y produce a la vez (Toffler, 1980). Al momento del consumo, el cliente también deja reseñas o realiza comentarios en lugares especializados, información que resulta de utilidad para generar insights que mejoren la experiencia en el servicio. 
En tercer lugar, se entiende al cliente como crítico, puesto que si la experiencia del cliente es negativa, el feedback y reseñas negativas que este brinde pueden ser de índole constructiva como destructiva. 
En cuarto lugar, se encuentra el cliente como pieza fundamental en el desarrollo de los productos y servicios. Los comentarios de los clientes pueden conducir al desarrollo de servicios innovadores apegados a las necesidades que los clientes indican. Para poder lograr perfeccionar el servicio y productos ofrecidos, es crucial el aporte de los clientes recurrentes o suscriptores del servicio, en el caso específico de las Administradoras de Fondos de Pensiones se refiere a los afiliados. 
En quinto lugar, el cliente como evaluador de la experiencia. Relacionado con los puntos anteriores, la mejor forma de mejorar la experiencia del cliente es tomando en consideración los comentarios de los clientes en esta materia, así se puede generar una diferencia de las otras empresas que constituyen la competencia existente en el mercado. 
Por último, se considera que el cliente puede ser un eventual embajador de la marca, en otras palabras promotores de la misma pudiendo generar recomendaciones, comentarios y reseñas positivas que promuevan el negocio. 
