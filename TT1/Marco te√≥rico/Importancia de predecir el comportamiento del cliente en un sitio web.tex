La predicción del comportamiento del cliente dentro de un entorno web se considera a la aplicación de técnicas y modelos analíticos para lograr predecir en cierta manera las posibles necesidades, acciones, preferencias y decisiones que un cliente pueda tomar mientras interactúa en alguna plataforma en línea o sitio web. En los últimos años, ha sido de gran importancia la predicción del comportamiento de los clientes para las empresas, gracias a esto buscan anticipar las necesidades y preferencias de sus clientes, pudiendo adaptar los productos y servicios para entregar una mayor satisfacción al cliente (Zheng, Thompson, Lam, Yoon y Gnanasambandam, 2013). La lealtad de los clientes representa un valor clave para las empresas, ya que un cliente leal seguirá consumiendo los productos y servicios de la empresa, por lo que si se mejora la experiencia del usuario, la satisfacción del cliente aumenta y esto genera un aumento en la ganancia de la empresa. 
Según Zheng, Thompson, Lam, Yoon y Gnanasambandam (2013), la predicción del comportamiento del cliente ayuda a las empresas a identificar oportunidades de mejora y mercado, además de ayudar a tomar decisiones informadas sobre estrategias de publicidad y marketing. El objetivo fundamental de predecir el comportamiento del cliente en un entorno web es lograr comprender y anticipar las acciones de los clientes con la meta de personalizar, mejorar la experiencia de usuario y poder aumentar la satisfacción y fidelidad de los clientes. 
Las predicciones pueden abarcar distintos aspectos del comportamiento de un cliente dentro de un canal web, a grandes rasgos existen 4 tipos de predicciones que se pueden realizar, están las predicciones de compras, donde mediante el análisis de patrones de navegación, su historial de compras, preferencias y características demográficas, gracias a esto se busca predecir las compras futuras de un cliente, se encuentra la predicción de clics, esta busca anticipar los enlaces o elementos con los cuales un cliente va a interactuar dentro de un sitio web, lo que busca mejorar la calidad de contenido que se encuentra desplegado y lograr mejorar la usabilidad del sitio web, también está presente la predicción de abandono de carrito, esta permite tomar acciones de recuperación o retención del cliente, se concentra en identificar aquellos clientes que agregan productos a un carrito de compra pero no finalizan el proceso de compra y por ultimo, esta la predicción de retención de clientes, esta busca predecir qué clientes están más cercanos a abandonar o terminar la relación existente con el sitio web, para poder generar e implementar estrategias para aumentar la fidelización y retención de estos clientes. 

