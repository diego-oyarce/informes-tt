La predicción del comportamiento del cliente dentro de un entorno web implica la aplicación de técnicas y modelos analíticos para anticipar, en cierta medida, las posibles necesidades, acciones, preferencias y decisiones que un cliente pueda tomar mientras interactúa en una plataforma en línea o sitio web. En los últimos años, la predicción del comportamiento de los clientes ha sido de gran importancia para las empresas, ya que les permite anticiparse a las necesidades y preferencias de sus clientes, adaptando así sus productos y servicios para brindar una mayor satisfacción al cliente \cite{behavior-prediction}.

La lealtad de los clientes es un valor clave para las empresas, ya que un cliente leal seguirá consumiendo los productos y servicios de la empresa. Por lo tanto, mejorar la experiencia del usuario aumenta la satisfacción del cliente, lo que a su vez genera un incremento en las ganancias de la empresa.

Según \cite{behavior-prediction}, la predicción del comportamiento del cliente ayuda a las empresas a identificar oportunidades de mejora y de mercado, además de respaldar la toma de decisiones informadas sobre estrategias de publicidad y marketing. El objetivo principal de predecir el comportamiento del cliente en un entorno web es comprender y anticipar las acciones de los clientes con el fin de personalizar y mejorar la experiencia del usuario, y así aumentar la satisfacción y fidelidad de los clientes.

Las predicciones pueden abarcar diferentes aspectos del comportamiento de un cliente dentro de un canal web. En términos generales, existen cuatro tipos de predicciones que se pueden realizar. En primer lugar, están las predicciones de compras, donde se analizan los patrones de navegación, el historial de compras, las preferencias y las características demográficas del cliente para predecir sus compras futuras. Luego, se encuentra la predicción de clics, que busca anticipar los enlaces o elementos con los cuales un cliente interactuará en un sitio web, con el objetivo de mejorar la calidad del contenido y la usabilidad del sitio. Además, se encuentra la predicción de abandono de carrito, que permite identificar a aquellos clientes que agregan productos a un carrito de compra pero no completan el proceso de compra, con el fin de tomar acciones de recuperación o retención del cliente. Por último, está la predicción de retención de clientes, que busca predecir qué clientes están más propensos a abandonar o finalizar su relación con el sitio web, para poder implementar estrategias que aumenten su fidelización y retención.