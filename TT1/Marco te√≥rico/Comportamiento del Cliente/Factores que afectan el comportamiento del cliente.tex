%Factores que influyen en el comportamiento del cliente en el canal web, tales como la usabilidad y el diseño del sitio web.
Lemon y Verhoef (2016) proponen que los principales factores que influyen en el comportamiento del usuario y su experiencia son sensoriales, afectivos, cognitivos, puntos de contacto y externos. Dentro de la experiencia sensorial se encuentra lo apreciable con alguno de los sentidos del cuerpo, tanto vista, olor, tacto, entre otros. Respecto de la experiencia afectiva, hay que tener en consideración la emocionalidad del cliente producto de la experiencia del producto o del servicio. Al hablar del aspecto cognitivo, este refiere de los pensamientos, creencias y/o actitudes que el cliente pueda tener respecto de la compañía, el producto o el servicio entregado. Sobre los puntos de contacto, estos hacen mención a las distintas maneras en las que el cliente y la compañía entran en contacto, tales como la publicidad, servicio al cliente, redes sociales o interacciones de tipo transaccional (Lemon y Verhoef, 2016). Por último, el factor externo cuya definición hace referencia a considerar el contexto actual, las condiciones socioeconómicas y otros factores que puedan afectar la experiencia del usuario que se encuentren fuera de control de la compañía. 
Dentro de los factores que pueden influir en el comportamiento de un cliente en el canal web están principalmente, la usabilidad y el diseño. Respecto a la usabilidad, esta depende de 7 características las que garantizan una buena experiencia del usuario. Según Sanchez (2011) la accesibilidad, legibilidad, navegabilidad, facilidad de aprendizaje, velocidad de utilización, eficiencia del usuario y tasas de error del canal web, influyen en la experiencia y posterior feedback que el usuario pueda brindar sobre el uso de los servicios. 
Por otro lado, el diseño del sitio web depende de 5 características para garantizar un buen contenido y estética para lograr que el usuario encuentre lo que busca en el menor tiempo posible, en otras palabras, eficiencia. El autor Walter Sanchez (2011) indica que el diseño debe de ser entendible, novedoso, comprensible, inteligente y atractivo, consiguiendo acercar los contenidos de mejor manera al usuario y logrando conseguir una navegación más intuitiva. Estos factores son de gran importancia para que el usuario pueda encontrar el contenido que busca en el menor tiempo posible y que la experiencia sea positiva al interactuar con la interfaz del sitio web. 
