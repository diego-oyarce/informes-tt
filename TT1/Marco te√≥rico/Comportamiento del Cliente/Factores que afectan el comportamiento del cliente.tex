% Factores que influyen en el comportamiento del cliente en el canal web, tales como la usabilidad y el diseño del sitio web.
\cite{lemon2016customer} proponen que los principales factores que influyen en el comportamiento del usuario y su experiencia son los sensoriales, afectivos, cognitivos, puntos de contacto y externos. La experiencia sensorial se refiere a los aspectos perceptibles por los sentidos del cuerpo, como la vista, el olfato y el tacto. En cuanto a la experiencia afectiva, se debe considerar la emocionalidad del cliente como resultado de la experiencia con el producto o servicio. En el aspecto cognitivo, se refiere a los pensamientos, creencias y actitudes que el cliente puede tener hacia la compañía, el producto o el servicio \cite{lemon2016customer}. Los puntos de contacto hacen referencia a las diferentes formas en que el cliente y la compañía interactúan, como la publicidad, el servicio al cliente, las redes sociales o las interacciones transaccionales. Por último, el factor externo se refiere al contexto actual, las condiciones socioeconómicas y otros factores que pueden afectar la experiencia del usuario y que están fuera del control de la compañía.

Dentro de los factores que pueden influir en el comportamiento de un cliente en el canal web, se encuentran principalmente la usabilidad y el diseño. En cuanto a la usabilidad, depende de siete características que garantizan una buena experiencia para el usuario. Según Sánchez \cite{Usabilidad=softw}, la accesibilidad, legibilidad, navegabilidad, facilidad de aprendizaje, velocidad de utilización, eficiencia del usuario y tasas de error del canal web influyen en la experiencia del usuario y en el feedback que este pueda brindar sobre el uso de los servicios.

Por otro lado, el diseño del sitio web depende de cinco características para garantizar un buen contenido y estética, y lograr que el usuario encuentre lo que busca en el menor tiempo posible, es decir, eficiencia. El autor Walter Sánchez \cite{Usabilidad=softw} indica que el diseño debe ser entendible, novedoso, comprensible, inteligente y atractivo, lo que permite acercar los contenidos de mejor manera al usuario y lograr una navegación más intuitiva. Estos factores son de gran importancia para que el usuario pueda encontrar el contenido que busca en el menor tiempo posible y tener una experiencia positiva al interactuar con la interfaz del sitio web.
