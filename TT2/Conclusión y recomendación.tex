\subsection{Conclusión}

El desarrollo de este proyecto de tesis ha abarcado una gama integral de procesos y análisis, comenzando con un estudio bibliométrico y extendiéndose a través del proceso ETL (Extracción, Transformación y Carga), el Análisis Exploratorio de Datos (EDA) y la implementación y evaluación de modelos predictivos.

El estudio bibliométrico proporcionó una base sólida y rigurosa, permitiendo una comprensión profunda del estado actual del campo, identificando las tendencias predominantes y las áreas menos exploradas en la literatura. Este análisis ayudó a orientar el enfoque del proyecto hacia áreas de investigación que son tanto innovadoras como de relevancia práctica.

En la fase de ETL, se trabajó meticulosamente para asegurar la calidad y la integridad de los datos. Este proceso fue crucial para preparar un conjunto de datos coherente y fiable, lo que sentó las bases para un análisis más detallado y preciso. La limpieza, transformación y enriquecimiento de los datos fueron pasos fundamentales que garantizaron la confiabilidad de las conclusiones y recomendaciones posteriores.

El Análisis Exploratorio de Datos (EDA) permitió una comprensión más profunda de los datos a mano. A través de este análisis, se identificaron patrones, anomalías y relaciones clave que influyeron significativamente en la selección y el ajuste de los modelos predictivos. Esta fase fue vital para la comprensión de las dinámicas subyacentes de los datos y para formular hipótesis informadas para la modelización.

Finalmente, la fase de modelado y pruebas fue el núcleo del proyecto. La aplicación de modelos predictivos, ajustados y evaluados cuidadosamente, proporcionó insights valiosos y respuestas a las preguntas de investigación planteadas. La evaluación de estos modelos, utilizando métricas como la precisión, el recall y la puntuación F1, reveló un rendimiento moderado, indicando áreas para futuras mejoras y ajustes. Esta fase no solo validó el enfoque metodológico empleado, sino que también abrió caminos para futuras investigaciones y desarrollos.

En conjunto, este proyecto ha demostrado la importancia de un enfoque metódico y bien estructurado en la investigación. Desde la recopilación inicial de datos hasta el análisis final y la interpretación de los resultados, cada etapa ha sido crucial para construir un entendimiento completo del tema en cuestión. Las lecciones aprendidas y los conocimientos adquiridos durante este proyecto no solo son valiosos para este campo de estudio, sino que también proporcionan un marco para futuras investigaciones y desarrollos en áreas relacionadas.


\subsection{Recomendaciones}

A través del desarrollo de este proyecto, se han identificado varias áreas clave en las que se pueden realizar mejoras y avances. Estas recomendaciones están orientadas a optimizar el modelo actual y a expandir el alcance de la investigación futura:

\subsubsection{Optimización del Modelo Actual:}
\begin{itemize}
    \item \textbf{Aumentar el Conjunto de Datos:} Un mayor volumen de datos puede mejorar la capacidad del modelo para aprender y generalizar. Esto es especialmente relevante en modelos basados en aprendizaje profundo que requieren grandes cantidades de datos para un rendimiento óptimo.
    \item \textbf{Exploración de Hiperparámetros:} Ajustar los hiperparámetros del modelo, como la tasa de aprendizaje, el número de capas ocultas y la cantidad de nodos en cada capa, puede resultar en mejoras significativas en el rendimiento.
    \item \textbf{Incorporación de Regularización:} Implementar técnicas de regularización como Dropout o L1/L2 puede ayudar a prevenir el sobreajuste y mejorar la generalización del modelo en datos no vistos.
\end{itemize}

\subsubsection{Experimentación con Diferentes Arquitecturas de Modelos:}
\begin{itemize}
    \item \textbf{Modelos de Aprendizaje Profundo Avanzados:} Probar con arquitecturas más complejas, como Redes Neuronales Recurrentes (RNN), Transformers o modelos basados en atención, podría capturar mejor las relaciones temporales y contextuales en los datos.
    \item \textbf{Enfoque Híbrido:} Combinar modelos de aprendizaje automático tradicionales con técnicas de aprendizaje profundo puede proporcionar una visión más holística y mejorar la precisión de las predicciones.
\end{itemize}

\subsubsection{Análisis de Características Adicionales:}
\begin{itemize}
    \item \textbf{Ingeniería de Características:} Explorar nuevas características derivadas de los datos existentes puede revelar patrones ocultos y mejorar la capacidad predictiva del modelo.
    \item \textbf{Integración de Datos Externos:} Incorporar datos externos, como tendencias de mercado o indicadores económicos, podría proporcionar contextos adicionales que mejoren la precisión predictiva.
\end{itemize}

\subsubsection{Evaluación Continuada y Ajuste del Modelo:}
\begin{itemize}
    \item \textbf{Monitoreo de Desempeño en Tiempo Real:} Establecer un sistema de monitoreo para evaluar el desempeño del modelo en tiempo real puede ayudar a identificar rápidamente áreas de mejora y ajustar el modelo de manera proactiva.
    \item \textbf{Actualización de Datos y Modelos:} Regularmente actualizar el modelo con nuevos datos para mantener su relevancia y precisión frente a patrones cambiantes y tendencias emergentes.
\end{itemize}

\subsubsection{Investigaciones Futuras:}
\begin{itemize}
    \item \textbf{Estudio de Casos Específicos:} Investigar en profundidad casos específicos donde el modelo no se desempeñó según lo esperado puede proporcionar insights valiosos para futuras mejoras.
    \item \textbf{Expansión a Nuevos Dominios:} Aplicar el modelo a diferentes conjuntos de datos o en distintos contextos podría ayudar a evaluar su robustez y adaptabilidad.
\end{itemize}

Estas recomendaciones están destinadas a guiar los esfuerzos futuros hacia la mejora continua del modelo y la expansión del conocimiento en este campo de estudio. La implementación exitosa de estas sugerencias requiere un enfoque iterativo y adaptable, asegurando que el modelo se mantenga relevante y efectivo frente a las cambiantes dinámicas de datos y requerimientos del mundo real.