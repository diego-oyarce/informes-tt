Para comprender la experiencia y el comportamiento del cliente en un canal web, es importante reconocer la existencia del customer journey, el cual describe las distintas etapas por las que un cliente pasa al consumir un producto o servicio. Según \cite{lemon2016customer}, estas etapas incluyen la conciencia, investigación, consideración, compra, uso y evaluación. La etapa de conciencia refiere a la identificación de una necesidad o problema que debe ser resuelto, mientras que la investigación implica la búsqueda de información por parte del cliente para encontrar posibles soluciones y comparar entre diferentes opciones disponibles. Luego, en la etapa de consideración, el cliente evalúa las alternativas y elige la que mejor se adapte a sus necesidades, lo que lleva a la etapa de compra, donde se realiza la contratación o adquisición del servicio seleccionado. Posteriormente, viene la etapa de uso, en la cual el cliente experimenta y evalúa la calidad, funcionalidad y experiencia del servicio. Por último, se encuentra la etapa de evaluación, en la cual el cliente emite un feedback voluntario, tanto positivo como negativo, sobre su experiencia satisfactoria o insatisfactoria. En resumen, las opciones disponibles en el canal web buscan hacer del customer journey una experiencia eficiente y agradable.

Para acceder al canal web de AFP Capital, es necesario ser afiliado y contar con una cuenta privada personal que incluya el RUT y contraseña. Una vez ingresado al canal web privado, los afiliados tienen a su disposición diversas opciones para satisfacer sus necesidades. Estas incluyen revisión del pago o no de la cotización mensual, la obtención de certificados de cotizaciones, afiliación, antecedentes previsionales y traspaso de fondos, así como certificados tributarios. Además, se pueden obtener certificados generales, como de residencia, suscripción de ahorro previsional voluntario (APV), cuenta 2, remuneraciones imponibles, periodos no cotizados y trabajo pesado. En el caso de afiliados pensionados, también se pueden obtener certificados de asignación familiar, calidad de pensionado, pensiones pagadas, pensión en trámite, ingreso base y comprobante de pago de pensión. Además, es posible acceder a la cartola en línea. El canal web privado permite realizar el ahorro obligatorio y voluntario, inversiones, depósitos directos, consultar planillas de pagos y ver las comisiones cobradas como afiliado. También ofrece la opción de verificar el fondo de pensiones, los tipos de fondos disponibles (A, B, C, D, E) y sus porcentajes de rentabilidad, así como realizar cambios de fondo de pensiones y acceder a educación previsional. Además, se brinda la posibilidad de realizar giros en cuentas personales, acceder a rescates financieros y tramitar la pensión.