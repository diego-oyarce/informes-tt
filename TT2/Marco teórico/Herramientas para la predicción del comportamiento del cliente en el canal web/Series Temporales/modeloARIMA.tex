El modelo ARIMA, por sobre otros modelos de series temporales, se centra en describir las autocorrelaciones que existen entre los datos \cite{forecast-time-series-arima}.

\begin{itemize}
    \item \textbf{Estacionariedad:} Una serie de tiempo estacionaria, se presenta cuando sus propiedades no dependen del momento en el que fue registrada la observación. Por lo que, las series de tiempo que presentan tendencias o patrones estacionales son series de tiempo no estacionarias, ya que las tendencias y las distintas estaciones de tiempo pueden afectar la serie de tiempo en varias ocasiones \cite{forecast-time-series-arima}. 
    
    Las series de tiempo estacionarias por lo general son series de ruido blanco, ya que no muestran autocorrelación entre los datos y tampoco presentan patrones predecibles a lo largo del tiempo. 

    \item \textbf{Diferenciación:} Una manera de poder cambiar una seria de tiempo no estacionaria en una estacionaria, es aplicar la diferenciación, esto se refiera a calcular la diferencia entre observaciones consecutivas \cite{forecast-time-series-arima}.
    
    Una de las transformaciones más ocupada para estabilizar la varianza de los datos son los logaritmos. Por otro lado, la diferenciación estabiliza el promedio de la seria de tiempo debido a que es capaz de reducir o remover los distintos cambios que se puedan presentar en la serie de tiempo, tales como las tendencias y patrones estacionales.
    
    
\end{itemize}