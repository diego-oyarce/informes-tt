En la era digital, los modelos de predicción de comportamiento del cliente son un recurso fundamental para las empresas que buscan tomar decisiones informadas y personalizar sus estrategias. Este subcapítulo del informe se adentrará en los diversos modelos utilizados para anticipar las acciones y preferencias de los clientes, destacando su relevancia en la toma de decisiones estratégicas y la mejora de la experiencia del usuario en el entorno online.

\begin{itemize}
    \item  \subsubsection{Modelos de regresión}
    \noindent
    La regresión logística corresponde a un algoritmo de aprendizaje automático supervisado que es empleado para resolver problemas de clasificación. Si bien, su nombre contiene “regresión”, en realidad corresponde a un método de clasificación.

    Se da uso a la regresión logística cuando la variable de respuesta o variable objetivo es categórica. En lugar de predecir un valor numérico como en la regresión lineal, la regresión logística estima la probabilidad de que una observación pertenezca a una categoría específica.

    Los modelos de regresión logística se basan en la función logística, también conocida como función sigmoide, que mapea cualquier valor real a un rango entre 0 y 1. La función sigmoide tiene la siguiente forma matemática:

    \begin{equation*}
        f(z) = \frac{1}{(1 + e^{-z})}
    \end{equation*}

    En la regresión logística, se ajusta un modelo lineal a los datos de entrada y se aplica la función sigmoide al resultado para obtener la probabilidad de pertenencia a una clase. La ecuación del modelo se expresa como:

    \begin{equation*}
        p(y=1|x) = \frac{1}{(1 + e^{(-(b0 + b1x1 + b2x2 + ... + bn*xn))})}
    \end{equation*}

    Donde:

    p(y=1|x) es la probabilidad condicional de que la variable de respuesta sea igual a 1 dada la entrada x.

    b0, b1, b2, ..., bn son los coeficientes del modelo que se ajustan durante el proceso de entrenamiento.

    x1, x2, ..., xn son los valores de las variables de entrada.

    El proceso de ajuste de la regresión logística implica encontrar los mejores valores para los coeficientes del modelo con la finalidad de maximizar la verosimilitud de los datos observados. Esto se puede hacer mediante métodos numéricos como la maximización de la función de verosimilitud o mediante algoritmos de optimización como el gradiente descendente.

    Una vez entrenado el modelo, se puede utilizar para hacer predicciones clasificando nuevas observaciones según la probabilidad estimada. Por ejemplo, si la probabilidad estimada de pertenencia a una clase es superior a un umbral (generalmente 0.5), se clasificará como perteneciente a esa clase.

    Para nuestro caso en particular, puede ser utilizado el modelo de regresión logística para predecir el comportamiento de usuarios en un canal web, para ello se necesitaría tener datos históricos que contengan información relevante sobre el comportamiento pasado de los usuarios y las variables predictoras asociadas. Estas variables predictoras pueden incluir características demográficas, patrones de uso del sitio web o aplicación, historial de compras, interacciones anteriores, entre otros.

    Una vez que se tienen los datos y las variables predictoras, se puede entrenar un modelo de regresión logística utilizando técnicas de ajuste como la maximización de la verosimilitud o el gradiente descendente. Una vez entrenado el modelo, puede ser utilizado para predecir el comportamiento futuro de los usuarios en función de nuevas observaciones o datos entrantes.

    Es importante tener en consideración que la calidad de las predicciones dependerá de la calidad de los datos utilizados para entrenar el modelo y de la selección adecuada de las variables predictoras. Además, es fundamental realizar una validación adecuada del modelo utilizando técnicas como la validación cruzada o la separación de conjuntos de entrenamiento y prueba para evaluar su rendimiento y generalización en datos no vistos.
    
    \begin{itemize}
        \item \textbf{Ventajas de los modelos de regresión logística}
            \begin{itemize}
                \item \textbf{Interpretación de resultados:} La regresión logística proporciona coeficientes que indican la dirección y la magnitud de la relación entre las variables predictoras y la variable de respuesta. Esto permite interpretar el efecto relativo de cada variable en la probabilidad de pertenecer a una clase específica.
                \item \textbf{Manejo de variables independientes categóricas:} La regresión logística puede manejar tanto variables independientes continuas como categóricas. Incluso puede manejar variables categóricas con más de dos categorías mediante técnicas como la codificación de variables ficticias.
                \item \textbf{Estimación de probabilidades:} La regresión logística estima la probabilidad de pertenencia a una clase específica en lugar de simplemente clasificar observaciones en categorías. Esto es útil cuando se necesita una medida de certeza o riesgo asociado con la clasificación.
                \item \textbf{Buena capacidad de generalización:} La regresión logística puede funcionar bien con conjuntos de datos pequeños o moderados, y es menos propensa al sobreajuste en comparación con otros algoritmos más complejos. Esto la hace adecuada para aplicaciones con muestras limitadas.   
            \end{itemize}
        \item \textbf{Desventajas de los modelos de regresión logística}
            \begin{itemize}
                \item \textbf{Linealidad de la relación:} La regresión logística asume una relación lineal entre las variables predictoras y la probabilidad logarítmica de la variable de respuesta. Si existe una relación no lineal, la regresión logística puede no ajustarse adecuadamente o requerir transformaciones adicionales de las variables.
                \item \textbf{Sensible a valores atípicos y datos faltantes:} Los valores atípicos o datos faltantes pueden afectar negativamente el rendimiento de la regresión logística. Es necesario manejarlos adecuadamente para evitar sesgos o imprecisiones en los resultados.
                \item \textbf{Suposición de independencia:} La regresión logística asume que las observaciones son independientes entre sí. Si hay dependencias o correlaciones entre las observaciones, la precisión de los resultados puede verse comprometida.
                \item \textbf{No apto para problemas no lineales:} Si existe una relación compleja y no lineal entre las variables predictoras y la variable de respuesta, la regresión logística puede no ser el modelo más adecuado. En tales casos, se pueden requerir técnicas más avanzadas, como modelos no lineales o de aprendizaje profundo.
            \end{itemize}
    \end{itemize}

\end{itemize}
