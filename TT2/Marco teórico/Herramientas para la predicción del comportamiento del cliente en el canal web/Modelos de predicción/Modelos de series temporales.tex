\\
Los modelos de series temporales son técnicas utilizadas para analizar y predecir datos secuenciales que están organizados en función del tiempo. En una serie temporal, los datos se registran en intervalos regulares (como horas, días, meses, etc.) y cada punto de datos está asociado con una marca de tiempo.

El objetivo principal de los modelos de series temporales es comprender y capturar los patrones, tendencias y estacionalidad en los datos a lo largo del tiempo, y utilizar esta información para hacer predicciones futuras. Estos modelos son ampliamente utilizados en diversos campos, como la economía, las finanzas, la meteorología, la demanda de productos, la planificación de inventario y más.

Los modelos de series temporales se basan en la suposición de que los datos pasados pueden proporcionar información útil para predecir el futuro. Algunos de los modelos más comunes utilizados en el análisis de series temporales son:
\begin{itemize}
    \item \textbf{Media móvil (MA):} Este modelo estima el valor futuro de la serie temporal en función de un promedio de los errores pasados. Se utiliza para capturar patrones aleatorios o no sistemáticos en los datos.
    \item \textbf{Autoregresión (AR):} Este modelo estima el valor futuro de la serie temporal en función de valores pasados de la propia serie. Se utiliza para capturar la dependencia de la serie en sí misma a lo largo del tiempo.
    \item \textbf{Autoregresión de media móvil (ARMA):} Este modelo combina los enfoques AR y MA para capturar tanto la dependencia de la serie en sí misma como los patrones aleatorios.
    \item \textbf{Autoregresión integrada de media móvil (ARIMA):} Este modelo amplía el modelo ARMA al considerar también las diferencias entre los valores de la serie temporal. Se utiliza para capturar tendencias y estacionalidad en los datos.
\end{itemize}

Además de estos modelos clásicos, también se utilizan enfoques más avanzados, como los modelos de espacio de estados, los modelos de suavizado exponencial y los modelos de redes neuronales recurrentes (RNN), que pueden capturar relaciones más complejas y no lineales en los datos de series temporales.

Es importante destacar que el análisis de series temporales requiere un enfoque cuidadoso para la selección del modelo, la identificación de patrones y la evaluación de la precisión de las predicciones. Además, se deben tener en cuenta factores como la estacionalidad, la estacionariedad de la serie y la presencia de datos faltantes o valores atípicos para obtener resultados confiables.

\begin{itemize}
    \item \textbf{Ventajas de los modelos de series temporales}
    \begin{itemize}
        \item \textbf{Captura de patrones temporales:} Los modelos de series temporales pueden capturar patrones, tendencias y estacionalidad en los datos a lo largo del tiempo. Esto permite comprender mejor la dinámica de los datos y hacer predicciones más precisas.
        \item \textbf{Predicciones a corto plazo:} Los modelos de series temporales son adecuados para hacer predicciones a corto plazo, ya que utilizan la información histórica para predecir los valores futuros. Esto es especialmente útil en aplicaciones donde se necesita anticipar eventos próximos, como demanda de productos o pronóstico del clima.
        \item \textbf{Utilización de datos secuenciales:} Los modelos de series temporales aprovechan la estructura secuencial de los datos y utilizan la información de los puntos anteriores para hacer predicciones en el siguiente punto. Esto permite tener en cuenta la dependencia temporal en los datos y obtener resultados más precisos.
        \item \textbf{Flexibilidad en la elección del modelo:} Existen diferentes tipos de modelos de series temporales que se pueden utilizar según la naturaleza de los datos y los patrones presentes. Esto proporciona flexibilidad para seleccionar el modelo más adecuado para el problema específico.
    \end{itemize}
    \item \textbf{Desventajas de los modelos de series temporales}
    \begin{itemize}
        \item \textbf{Sensibilidad a datos faltantes o valores atípicos:} Los modelos de series temporales pueden verse afectados negativamente por la presencia de datos faltantes o valores atípicos. Estos pueden distorsionar los patrones y afectar la precisión de las predicciones.
        \item \textbf{Dificultad con tendencias no lineales:} Los modelos de series temporales asumen a menudo que las relaciones son lineales o pueden ser capturadas por modelos lineales. Si hay tendencias no lineales en los datos, los modelos lineales pueden no ajustarse adecuadamente y se pueden requerir enfoques más avanzados.
        \item \textbf{Necesidad de datos históricos adecuados:} Los modelos de series temporales requieren una cantidad suficiente de datos históricos para hacer predicciones precisas. En ausencia de datos suficientes, los modelos pueden tener dificultades para capturar patrones y generar resultados confiables.
        \item \textbf{Problemas con cambios estructurales:} Si hay cambios estructurales significativos en los datos de series temporales (por ejemplo, cambios en la estacionalidad o en los patrones), los modelos de series temporales pueden tener dificultades para adaptarse y pueden requerir ajustes manuales.
    \end{itemize}
\end{itemize}