Se llevó a cabo la implementación de una API usando Flask, una librería de Python para crear aplicaciones web, y TensorFlow, una librería para el aprendizaje automático. La API está diseñada para realizar clasificación de secuencias, utilizando un modelo previamente entrenado y dos diccionarios JSON para el mapeo de datos.

\section{Descripción de los componentes}

Estos componentes trabajan juntos para permitir que la API reciba datos de entrada, los procese utilizando el modelo de clasificación y los mapeos, y luego devuelva una predicción con su probabilidad asociada.

\subsubsection{Modelo de Clasificación con TensorFlow:}

\begin{itemize}
    \item \textbf{Uso de TensorFlow Keras:} El modelo está construido y entrenado utilizando TensorFlow, específicamente su API de alto nivel, Keras. TensorFlow es una plataforma de código abierto para aprendizaje automático, y Keras facilita la creación y entrenamiento de modelos de aprendizaje profundo.
    \item \textbf{Archivo .h5:} Este formato de archivo es específico para modelos de Keras y almacena la arquitectura del modelo, los pesos de las neuronas, y la configuración de entrenamiento. El modelo .h5 que se carga es un modelo pre entrenado y listo para realizar predicciones.
\end{itemize}

\subsubsection{Diccionarios JSON para Mapeo de Datos:}

\begin{itemize}
    \item \textbf{Archivos method\_to\_int.json y channel\_to\_int.json:} Estos archivos contienen mapeos en formato JSON, que se utilizan para convertir los datos categóricos de entrada (métodos y canales) en valores numéricos. Esto es necesario porque los modelos de aprendizaje automático requieren entradas numéricas.
    \item \textbf{Función de los Mapeos:} Cada método y canal se asigna a un número entero único. Por ejemplo, un método específico podría mapearse al número 3, y un canal al número 5. Estos mapeos son esenciales para procesar las entradas de manera que sean coherentes con el formato de datos usado para entrenar el modelo.
\end{itemize}

\subsubsection{API de Flask}

\begin{itemize}
    \item \textbf{Flask como Marco de Trabajo:} Flask es un microframework para Python que permite desarrollar aplicaciones web de manera sencilla. En este caso, se utiliza para crear un servidor web que puede manejar solicitudes HTTP.
    \item \textbf{Ruta /predict:} Esta es una ruta definida en la aplicación Flask. Cuando se recibe una solicitud POST a esta ruta, la API procesa la solicitud para realizar una predicción utilizando el modelo cargado.
    \item \textbf{Manejo de Solicitudes y Respuestas:} Flask se encarga de interpretar las solicitudes HTTP, extraer los datos JSON enviados, y luego devolver una respuesta en formato JSON después de realizar la predicción.
\end{itemize}