El proceso de creación de nuestra API, comienza con la preparación del entorno de desarrollo. Utilizamos Python, un lenguaje de programación popular, y varias de sus bibliotecas especializadas. Primero, importamos Flask, una herramienta sencilla pero poderosa para crear aplicaciones web. Esto nos permite recibir y responder a solicitudes a través de internet. También importamos TensorFlow, una biblioteca avanzada para trabajar con aprendizaje automático, que nos ayudará a usar un modelo de clasificación, es decir, un programa entrenado para categorizar datos.

Nuestro modelo de clasificación está almacenado en un archivo .h5, un formato que conserva toda la estructura y los datos del modelo. Lo cargamos usando TensorFlow, lo que significa que preparamos el modelo para que esté listo para hacer predicciones.

Además del modelo, trabajamos con dos archivos JSON, method\_to\_int.json y channel\_to\_int.json. JSON es un formato de archivo que almacena información de manera estructurada y es fácil de leer tanto para humanos como para máquinas. Estos archivos contienen mapeos, que son como diccionarios que nos ayudan a convertir texto a números, ya que nuestro modelo no puede procesar texto directamente. Por ejemplo, si tenemos un método llamado \textquotedblleft análisis\textquotedblright y un canal llamado \textquotedblleft digital\textquotedblright, estos archivos nos dirán a qué número corresponde cada uno.

Luego, en nuestra aplicación Flask, creamos una \textquotedblleft ruta\textquotedblright especial, \/predict. Cuando alguien envía una solicitud a esta dirección, nuestra API recibe datos en formato JSON, los procesa usando los mapeos que mencionamos antes, y ajusta la longitud de estos datos para que coincida con lo que el modelo espera, usando algo llamado pad\_sequences.

Ahora, con los datos procesados, los introducimos en nuestro modelo de clasificación. El modelo analiza estos datos y nos da una predicción, que es su mejor conjetura sobre a qué categoría pertenecen los datos. Por ejemplo, puede predecir si un texto es sobre tecnología, deportes, etc., y con qué grado de confianza hace esta predicción.

Finalmente, nuestra API responde a quien envió la solicitud, devolviendo la categoría predicha y la confianza del modelo en esa predicción, todo en formato JSON para que sea fácil de entender y usar.

Para asegurarnos de que todo funcione correctamente, ejecutamos nuestra aplicación Flask en un modo especial de \textquotedblleft depuración\textquotedblright, que nos ayuda a identificar y solucionar problemas rápidamente.