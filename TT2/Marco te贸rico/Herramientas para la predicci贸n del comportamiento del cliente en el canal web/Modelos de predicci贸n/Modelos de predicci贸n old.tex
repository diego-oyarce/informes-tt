%Modelos de predicción del comportamiento del cliente en el canal web
Existen diferentes modelos empleados para realizar predicciones del comportamiento de un usuario en un canal web, el empleo de dichos modelos se encuentran explicados a continuación:

\subsubsection{Modelos de regresión}
El modelo de regresión es empleado para la predicción de variables, la regresión logística estima la probabilidad de que suceda un evento basándose en un conjunto de datos. Existen 3 tipos de modelos de regresión, los cuales corresponden a los modelos de regresión logística binaria, regresión logística multinomial y regresión logística ordinal.

Siendo el Modelo de regresión lógica binaria es empleado para predecir comportamientos de variables que tienen un comportamiento dicotómico, es decir, que solo cuentan con dos resultados posibles, como ejemplo podemos mencionar a la clasificación de correo electrónico si es spam o no lo es, si una opción es verdadero o falso, entre otros ejemplos. Dentro de la regresión logística es el más utilizado y en general, corresponde a uno de los clasificadores más comunes para la clasificación binaria.

El Modelo de regresión logística multinomial es empleado para predecir cuando la variable dependiente cuenta con tres o más resultados posibles, cabe recalcar que los valores a predecir no se encuentran ordenados.

Finalmente, el Modelo de regresión logística ordinal es utilizado cuando la variable de respuesta tiene tres o más resultados posibles, pero a diferencia del modelo de regresión logística multinomial, los valores empleados si poseen un orden definido.

\subsubsection{Ventajas de los modelos de regresión}

\begin{itemize}
    \item La implementación del modelo es más sencillo comparado con otros modelos.
    \item Interpretación de los resultados relativamente sencilla.
    \item Simplificación de problemas.
    \item Existencia de documentación respecto a los modelos.
\end{itemize}

\subsubsection{Desventajas de los modelos de regresión}

\begin{itemize}
    \item Propensos a sobreentrenarse.
    \item Alta sensibilidad a valores atípicos.
    \item Son propensos a realizar suposiciones.
\end{itemize}

\subsubsection{Modelos de recomendación}
Los modelos de recomendación corresponden a una subclase de aprendizaje automático que es utilizado para clasificar o valorar productos y usuarios. En modo de resumen, es un sistema de recomendación en un sistema que predice las valoraciones que un usuario puede dar a un producto y/o servicio. Este modelo es empleado en grandes empresas como Google, Amazon, Instagram, Spotify, entre otros.

Este modelo se puede clasificar en sistemas de filtrado colaborativo, sistemas basados en contenido o sistema de recomendación híbrido.

Los sistemas de filtrado colaborativo corresponden al proceso en el cual se predicen los intereses de un usuario, esto se realiza al identificar las preferencias e información de varios usuarios, realizando una búsqueda de patrones utilizando de filtrado de información.

Los sistemas basados en contenido se enfocan en generar recomendaciones basadas en las preferencias y los perfiles de usuario. El modelo trata de hacer coincidir a los usuarios con elementos o productos que les hayan gustado anteriormente. Un punto a destacar de este sistema es que se basa en los productos destacados por el usuario objetivo, en cambio, los otros modelos se basan en el usuario y en otros usuarios que utilizan la plataforma.

Para terminar con los modelos de recomendación, los sistemas de recomendación híbridos se encuentran diseñados para utilizar diferentes fuentes de información para generar las recomendaciones. 

\subsubsection{Ventajas de los modelos de recomendación}
\begin{itemize}
    \item Aumenta la participación de los usuarios.
    \item Recomendación de elementos acorde a los gustos de los usuarios.
    \item Poseen la capacidad de automatizar sus sugerencias.
\end{itemize}

\subsubsection{Desventajas de los modelos de recomendación}
\begin{itemize}
    \item Propenso a presentar sesgo y falta de diversificación.
    \item Autorización de terceros por privacidad (dado que utilizan datos personales).
    \item Presentan problemas de arranque en frío.
    \item Difíciles de interpretar.
\end{itemize}

\subsubsection{Modelos de series temporales}
 
Los modelos de series temporales corresponden a un proceso en el cual son utilizados datos pasados con la finalidad de predecir acontecimientos futuros. Estos modelos analizan tendencias y patrones en los datos para extraer información para realizar predicciones sobre valores futuros, este tipo de modelos son utilizados en el ámbito financiero para predecir ventas o cotizaciones, otro de los campos en los que es empleado este modelo es en el científico, enfocándose en predecir los patrones meteorológicos. 

\subsubsection{Ventajas de los modelos de series temporales}
\begin{itemize}
    \item Capaces de capturar patrones.
    \item Útiles para predecir a corto plazo.
    \item Proporcionan facilidades para la interpretación de los resultados.
\end{itemize}

\subsubsection{Desventajas de los modelos de series temporales}
\begin{itemize}
    \item No son eficientes para predecir a largo plazo.
    \item Son muy sensibles a los datos atípicos.
    \item Requieren información consistente.
\end{itemize}

\subsubsection{Modelos de aprendizaje automático}
Los modelos de aprendizaje automático corresponden a programas capaces de identificar patrones o tomar decisiones. Estos pueden ser entrenados para realizar predicciones, identificar objetos e imágenes, también puede ser empleado para predecir comportamientos de usuarios.

\subsubsection{Ventajas de los modelos de aprendizaje automático}
\begin{itemize}
    \item Permiten trabajar con grandes volúmenes de datos.
    \item Permite la automatización de tareas.
    \item Mejoran a medida que son utilizados.
\end{itemize}

\subsubsection{Desventajas de los modelos de aprendizaje automático}
\begin{itemize}
    \item Vulnerables a ataques de terceros.
    \item Requieren de una capacidad significativa de recursos.
    \item Algunos modelos entregan resultados complicados de interpretar.
\end{itemize}

\subsubsection{Modelos de análisis de sentimientos}
Como dice el nombre, el modelo de análisis de sentimientos permite procesar y analizar información a tiempo real, es utilizado principalmente en las redes sociales para analizar cómo ve la gente un producto nuevo y lo que no les gusta a sus clientes.

\subsubsection{Ventajas de los modelos de análisis de sentimientos}
\begin{itemize}
    \item El modelo es escalable.
    \item Permite la automatización de los procesos del modelo.
    \item Permite segmentar y personalizar los métodos de análisis.
\end{itemize}

\subsubsection{Desventajas de los modelos de análisis de sentimientos}
\begin{itemize}
    \item Al modelo se le dificulta identificar el sarcasmo.
    \item Posee limitaciones al momento de identificar sentimientos complejos.
    \item Los resultados del modelo pueden verse afectados por sesgos en los datos de entrenamiento.
\end{itemize}

\subsubsection{Modelos de detección de anomalías}
Los modelos de detección de anomalías corresponden a modelos de machine learning enfocados en detectar actividades extrañas en la información que está analizando, con esto queremos decir que detecta anomalías.

Existen diferentes tipos de métodos para detectar anomalías con machine learning, entre ellos se encuentran los modelos supervisados, sin supervisión y semi supervisado.

Para el método de modelos supervisados, el entrenamiento del modelo consta de dos variables de entrenamiento: normal y anormal. El modelo utiliza los ejemplos para detectar patrones y detectar anomalías en los datos proporcionados. Es importante destacar que, en el entrenamiento supervisado es fundamental asegurar la calidad de los datos con el cual será entrenado el modelo.

Para el método de modelos sin supervisión, es común emplear redes neuronales para implementar el modelo, dado que al utilizar redes neuronales se disminuye la carga de trabajo manual, con esto queremos decir que no es necesario preparar los datos de antemano. Un punto negativo es que la dificultad de implementar una red neuronal es alta.

Finalmente, para el modelo semi supervisado es una combinación de los dos métodos anteriores, permitiendo contar con las ventajas de ambos métodos. Al ser supervisado durante su entrenamiento, es posible tener un mayor control del tipo de patrones que aprender el modelo.

\subsubsection{Ventajas de los modelos de detección de anomalías}
\begin{itemize}
    \item El modelo permite automatizar la detección de anomalías.
    \item Posee gran adaptabilidad a datos y entornos.
    \item Permite detectar de manera temprana las anomalías.
\end{itemize}

\subsubsection{Desventajas de los modelos de detección de anomalías}
\begin{itemize}
    \item Es difícil definir una anomalía.
    \item Requiere de un entrenamiento adecuado al modelo.
    \item Es muy propenso a entregar posibles falsos positivos y falsos negativos.
\end{itemize}

\subsubsection{Modelos de atribución}
Los modelos de atribución permiten predecir el recorrido que los clientes seguirán al momento de concretar una compra. Este recorrido puede contener las redes sociales, el uso del sitio web del vendedor, el correo electrónico, entre otros. Los modelos de atribución permiten determinar el impacto que tiene el uso de las acciones para el sistema de marketing. 

Existen varios modelos adicionales para el uso del marketing y la predicción de comportamiento, sin embargo, para este proyecto se limitó la búsqueda a los modelos mencionados anteriormente, los cuales se adaptan mejor a nuestras necesidades.

\subsubsection{Ventajas de los modelos de atribución}
\begin{itemize}
    \item Facilita el rastrear de mejor manera el paso a paso del cliente.
    \item Permiten mayor personalización de rastreo de los clientes.
\end{itemize}

\subsubsection{Desventajas de los modelos de atribución}
\begin{itemize}
    \item Poseen una mayor complejidad que los otros modelos.
    \item La interpretación de los resultados puede ser subjetiva.
    \item Poseen limitaciones en la medición del seguimiento.
\end{itemize}
