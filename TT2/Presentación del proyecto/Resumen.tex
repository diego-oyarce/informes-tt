El presente documento de Trabajo de Titulación resume exhaustivamente el proceso completo llevado a cabo para desarrollar un modelo de aprendizaje automático capaz de predecir el comportamiento de los usuarios en la nueva plataforma virtual de AFP Capital. Además de detallar la creación de este modelo, se enfoca en la relevancia de la predicción del comportamiento de los clientes y su valor estratégico para las empresas en la actualidad, destacando la importancia de obtener y manejar datos precisos sobre sus comportamientos.

El documento explora varios modelos de aprendizaje considerados como opciones viables, profundizando en la implementación detallada de tres modelos específicos y presentando los resultados obtenidos. Se destaca el proceso de selección del modelo más eficiente entre los evaluados, describiendo minuciosamente este modelo elegido y documentando las pruebas realizadas para demostrar su eficacia final.

Además del desarrollo del modelo, se aborda la creación de un servicio consumible que genera predicciones utilizando el modelo mencionado. Se describe la implementación de una API que facilita la generación de predicciones basadas en los datos de entrada necesarios para el modelo.

El proyecto concluye con la dockerización de todo el trabajo realizado, lo que permite su despliegue simplificado en la plataforma virtual de la empresa. Este enfoque busca garantizar la accesibilidad y la facilidad de implementación del servicio.

Finalmente, se incluyen recomendaciones para mejorar y continuar con el proyecto en caso de que la empresa desee seguir utilizando este modelo. Además, se presentan conclusiones y reflexiones del grupo de trabajo sobre el desarrollo y los hallazgos alcanzados durante el proyecto.

\textbf{Palabras clave:}  Afiliado, Administradora de Fondos de Pensiones, API (Application Programming Interfaces), EDA (Exploratory Data Analysis), Algoritmos de predicción, Algoritmos de clasificación, Modelos de predicción, ETL (Extract, Transform and Load), ARIMA (Modelo de Autorregresión integrada de media móvil), SARIMA (Modelo Estacional de Autorregresión integrada de media móvil), Redes Neuronales Artificiales (ANN), Redes LSTM (Long Short-Term Memory), Redes Neuronales Recurrentes (RNN).