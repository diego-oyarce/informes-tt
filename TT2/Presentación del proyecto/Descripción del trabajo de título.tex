El trabajo de titulación se basa en un proyecto empresarial que tiene como objetivo procesar los registros de navegación del sitio web para afiliados de AFP Capital. El propósito principal es detectar comportamientos de los clientes y sus preferencias de uso, con el fin de personalizar las futuras experiencias de navegación. La lectura de los registros se realizará extrayendo la información desde Kibana, una plataforma basada en ElasticSearch, que registra la información a través de diversas APIs utilizadas en el sitio web.

Los elementos fundamentales del proyecto incluyen el análisis exploratorio de datos, extracciones, transformaciones, cargas, modelos de predicción y detección de preferencias. El objetivo final es desarrollar un modelo capaz de predecir el comportamiento de los clientes en el canal web.