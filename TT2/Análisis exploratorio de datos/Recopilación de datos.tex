Los datos recopilados de los registros de navegación de los afiliados de AFP Capital constituyen una valiosa fuente de información para comprender el comportamiento y las preferencias de los usuarios en la plataforma web. Estos registros nos permiten analizar cómo interactúan los afiliados con los diferentes canales y métodos disponibles, así como realizar un seguimiento detallado de las fechas y horarios en que se llevan a cabo estas interacciones.

El dataset entregado para este proyecto cuenta con 2,331,023 registros de navegación de usuarios, lo cual proporciona una cantidad significativa de información para su análisis. Antes de utilizar estos datos, se realizó un proceso de anonimización para proteger la privacidad de los usuarios, específicamente modificando el campo del rut para no mostrar el dato original. De esta manera, se garantiza que los registros sean tratados de forma confidencial y segura.

Los cuatro campos que conforman el conjunto de datos son el rut, la fecha del evento, el método y el canal. El rut, que ha sido modificado, actúa como un identificador único para cada usuario y permite realizar análisis individuales sin revelar su identidad. La fecha del evento registra el momento exacto en que se llevó a cabo cada navegación, lo cual es crucial para identificar patrones y tendencias a lo largo del tiempo. El campo del método describe la interacción específica realizada por el usuario en el canal correspondiente, proporcionando información detallada sobre las acciones que realizan. Por último, el campo del canal indica el sitio o ambiente particular en el cual tuvo lugar cada interacción, lo que puede ser útil para comprender las preferencias de los usuarios en relación con los diferentes entornos disponibles.

Con respecto al preprocesamiento de datos realizado hasta la fecha, se ha seguido el proceso ETL (Extracción, Transformación y Carga) que se describe en detalle en el capítulo anterior. Este proceso implica extraer los datos de las fuentes de origen, transformarlos en un formato adecuado y cargarlos en un sistema de almacenamiento para su posterior análisis. Se utilizaron diversas herramientas para llevar a cabo estas tareas, asegurando la calidad y coherencia de los datos procesados.

Es crucial subrayar que los datos recopilados, aunque constituyen una fuente de información considerable sobre el comportamiento de los usuarios en la plataforma web, presentan ciertas limitaciones que deben ser tenidas en cuenta. La muestra de datos podría no representar de manera equitativa todos los segmentos de ingresos de los afiliados, lo que podría influir en los patrones de navegación observados. Es posible que ciertas tendencias o preferencias sean más comunes en los registros debido a una sobrerrepresentación de determinados grupos económicos en los datos disponibles. Por ende, al interpretar los resultados y al intentar generalizar las conclusiones obtenidas, es esencial considerar que los comportamientos detectados pueden no ser extrapolables a la totalidad de la base de usuarios. Se debe proceder con precaución al hacer inferencias, y sería beneficioso buscar estrategias para incorporar y analizar datos de otros segmentos para obtener una comprensión más holística y representativa del conjunto de afiliados.