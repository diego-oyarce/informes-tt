Los datos recopilados de los registros de navegación de los afiliados de AFP Capital constituyen una valiosa fuente de información para comprender el comportamiento y las preferencias de los usuarios en la plataforma web. Estos registros nos permiten analizar cómo interactúan los afiliados con los diferentes canales y métodos disponibles, así como realizar un seguimiento detallado de las fechas y horarios en que se llevan a cabo estas interacciones.

El dataset inicial entregado para este proyecto cuenta con 58,252 registros de navegación de usuarios, lo cual proporciona una cantidad significativa de información para su análisis. Antes de utilizar estos datos, se realizó un proceso de anonimización para proteger la privacidad de los usuarios, específicamente modificando el campo del rut para no mostrar el dato original. De esta manera, se garantiza que los registros sean tratados de forma confidencial y segura.

Los cuatro campos principales que conforman el conjunto de datos son el rut, la fecha del evento, el método y el canal. El rut, que ha sido modificado, actúa como un identificador único para cada usuario y permite realizar análisis individuales sin revelar su identidad. La fecha del evento registra el momento exacto en que se llevó a cabo cada navegación, lo cual es crucial para identificar patrones y tendencias a lo largo del tiempo. El campo del método describe la interacción específica realizada por el usuario en el canal correspondiente, proporcionando información detallada sobre las acciones que realizan. Por último, el campo del canal indica el sitio o ambiente particular en el cual tuvo lugar cada interacción, lo que puede ser útil para comprender las preferencias de los usuarios en relación con los diferentes entornos disponibles.

Con respecto al preprocesamiento de datos realizado hasta la fecha, se ha seguido el proceso ETL (Extracción, Transformación y Carga) que se describe en detalle en el capítulo anterior. Este proceso implica extraer los datos de las fuentes de origen, transformarlos en un formato adecuado y cargarlos en un sistema de almacenamiento para su posterior análisis. Se utilizaron diversas herramientas especializadas para llevar a cabo estas tareas, asegurando la calidad y coherencia de los datos procesados.

Es importante destacar que, si bien los datos recopilados ofrecen una valiosa perspectiva sobre el comportamiento de los usuarios en la plataforma web, es necesario tener en cuenta que existe un sesgo en la muestra de datos. En particular, los registros de navegación corresponden principalmente a afiliados con rentas altas. Esto implica que los usuarios con ingresos más altos, aquellos que cotizan por un valor elevado o el valor máximo, están sobrerrepresentados en la muestra. Por lo tanto, al interpretar y generalizar los resultados obtenidos, es fundamental tener en cuenta esta limitación y considerar posibles variaciones en el comportamiento de otros segmentos de usuarios.