%Breve descripción del objetivo del análisis exploratorio de datos.
%Explicación de la importancia de comprender los datos antes de construir el modelo de predicción de comportamiento de los clientes.
El análisis exploratorio de datos (EDA, por sus siglas en inglés, Exploratory Data Analysis) es una fase fundamental en la investigación y comprensión de un conjunto de datos. Como su nombre lo indica, el EDA tiene como objetivo explorar y examinar los datos de manera detallada, utilizando resúmenes numéricos y visuales, con el fin de descubrir patrones, tendencias y características no anticipadas. Es considerado uno de los primeros pasos en el proceso de análisis, ya que proporciona una visión general de los datos antes de realizar un análisis más profundo \cite{ruiz2022exploratorio}.

El enfoque principal del EDA radica en el uso de herramientas y técnicas visuales y gráficas para revelar información clave sobre los datos en estudio \cite{parra2002exploratorio}. Estas técnicas incluyen el diagrama de tallo y hoja, el diagrama de caja y bigotes, y el diagrama de dispersión, entre otros. Al aplicar estas técnicas de análisis gráfico, podemos obtener una comprensión más profunda de la distribución y estructura de los datos, así como identificar relaciones entre las variables de interés. Además, el EDA nos brinda la capacidad de detectar posibles errores o puntos extremos, como anomalías, que podrían afectar la calidad de los resultados del análisis.

Los beneficios clave del análisis exploratorio de datos son los siguientes:
\begin{itemize}
    \item \textbf{Conocer la distribución y estructura de los datos:} El EDA nos permite examinar la distribución de las variables y comprender cómo se organizan y dispersan los datos en el conjunto. Esto es fundamental para seleccionar las técnicas adecuadas de análisis estadístico y modelado.
    \item \textbf{Estudiar la relación entre variables:} Mediante el análisis de correlación y la visualización de patrones en los diagramas de dispersión, podemos explorar las relaciones entre las variables y comprender cómo interactúan entre sí. Esto nos brinda información valiosa para identificar posibles dependencias y tendencias en los datos.
    \item \textbf{Encontrar posibles errores y anomalías:} El EDA nos ayuda a identificar valores atípicos, datos faltantes u otros errores en los datos. Estas anomalías pueden tener un impacto significativo en los resultados del análisis, por lo que es importante detectarlas y tratarlas de manera adecuada.
\end{itemize}