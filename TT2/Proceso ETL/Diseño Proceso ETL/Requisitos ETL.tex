En esta etapa se definen los requisitos del proyecto, las fuentes de datos, los objetivos comerciales y del proceso ETL, las necesidades de análisis y los plazos para realizar el proceso. Estableciendo una base solida para el diseño y buen funcionamiento del proceso ETL.
\begin{itemize}
    \item \textbf{Fuente de datos:} La fuente de datos corresponde a un archivo .CSV que contiene información de la navegación web de los clientes en forma de Web Logs.
    \item \textbf{Objetivos comerciales:} Analizar el comportamiento de los clientes y sus preferencias de uso en un período igual o inferior a 6 meses, para poder predecir navegaciones futuras, y a partir de esto proporcionar atenciones personalizadas.
    \item \textbf{Objetivos proceso ETL:} Realizar las transformaciones necesarias para asegurar que el flujo de datos sea eficiente y preciso, a través de la limpieza de los datos, la normalización, la agregación, el filtrado, el enriquecimiento de datos, así como los cálculos y derivaciones necesarios.
    \item \textbf{Necesidades de análisis:} Realizar un análisis exploratorio de los datos entregados.
\end{itemize}
