En esta etapa, se lleva a cabo el diseño del modelo de datos objetivo para poder trabajar con él en el desarrollo de un modelo de aprendizaje automático. Para esto se considera lo siguiente:
\begin{itemize}
    \item  \textbf{Tabla(s) necesaria(s):} En este escenario, se trabaja exclusivamente con una tabla que contiene los registros de navegación, como se ha comentado en secciones anteriores.
    \item \textbf{Definición de columnas o atributos:} El DataFrame contendrá nueve columnas que describirán el registro de actividad del usuario en la plataforma. Dichas columnas son: rut, año, mes, día, hora, minuto, segundo, método y canal. Cada columna proporcionará información valiosa sobre el comportamiento del usuario durante la interacción.
    \item \textbf{Definición de tipos de datos:} Los datos que se utilizarán serán numéricos, ya que la mayoría de los modelos de aprendizaje automático trabajan eficazmente con este tipo de datos para realizar predicciones.
    \item \textbf{Gestión de valores faltantes:} En este contexto, las filas que contengan valores faltantes se eliminarán del conjunto de datos. La decisión de no utilizar otras técnicas para gestionar los valores faltantes, como reemplazarlos con el promedio, se basa en la naturaleza del trabajo. El equipo concluyó que reemplazar los valores faltantes podría alterar la representación del comportamiento real de un usuario, por lo que se prefiere mantener la integridad de los registros de actividad.
    \item \textbf{Gestión de datos categóricos:} Dado que la mayoría de los datos son categóricos, es fundamental abordarlos de manera adecuada. Se aplicará una "codificación de etiqueta", que asigna un valor numérico único a cada categoría en función de alguna lógica o preferencia. En este caso, se asignarán valores numéricos a las categorías en orden ascendente a medida que aparezcan en el conjunto de datos.
    \item \textbf{Creación del nuevo dataframe:} La creación del nuevo DataFrame se llevará a cabo en Google Colab utilizando Python, específicamente la biblioteca Pandas. Las transformaciones realizadas se profundizarán en la siguiente sección para detallar cómo se optimizan los datos para análisis y modelado subsiguientes.
\end{itemize}