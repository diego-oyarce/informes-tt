En esta etapa se determinan las fuentes de datos a ser usadas para el proyecto, incluyendo bases de datos, archivos .CSV y APIs. Esto además comprende la definición de la estructura, el formato y ubicación de cada fuente de datos dentro del proyecto.

La fuente de datos corresponde a un archivo .CSV que contiene información de la navegación web de los afiliados en forma de web logs. Los web logs registrados vienen con 4 campos, especificados a continuación:

\begin{itemize}
\item \textbf{rut}: Este atributo representa un identificador único por afiliado. Debido a las políticas de confidencialidad de la empresa, antes de entregar el archivo, se realiza un proceso de anonimización de este campo. En el archivo, este dato se visualiza como una cadena de caracteres que incluye letras (minúsculas y mayúsculas), números y otros caracteres.
\item \textbf{fecha evento}: Representa el momento exacto en que el usuario realiza la interacción en el canal web. Esto incluye día, mes, año, hora, minuto y segundo.
\item \textbf{metodo}: Este atributo representa la interacción realizada por el usuario en el canal web.
\item \textbf{canal}: Corresponde al canal web donde el afiliado realiza su interacción en la plataforma.
\end{itemize}

Para almacenar la fuente de datos se utiliza una estructura de carpetas, siendo las siguientes:

\begin{itemize}
\item \textbf{Input}: Dentro de esta carpeta se encontrarán los archivos.
\item \textbf{Intermediate}: Aquí se almacenará el archivo con la información preprocesada.
\item \textbf{Output}: Dentro de esta última carpeta se almacenará la información ya procesada y lista para ser usada.
\end{itemize}