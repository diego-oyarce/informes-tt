Durante esta etapa, se planifican las transformaciones necesarias para establecer una base sólida que respalde el proyecto. Estas incluyen pasos para limpiar, filtrar, combinar y enriquecer los datos.

La planificación de estas transformaciones es esencial para asegurar la calidad y la integridad de los datos utilizados en el proyecto. Se identifican tareas específicas alineadas con los objetivos y requisitos del proyecto y las necesidades comerciales.

Algunas transformaciones comunes incluyen \cite{etl-toolkit}:

\begin{itemize}
    \item \textbf{Limpieza de datos:} Se llevan a cabo tareas de limpieza para corregir errores, eliminar valores duplicados o inconsistentes y garantizar la coherencia de los datos. En este caso, la limpieza de datos se utilizará para estandarizar el formato y eliminar valores que puedan afectar de manera negativa el aprendizaje del modelo. Inicialmente, se eliminarán todas las filas que contengan valores nulos. Es importante mencionar que algunos modelos trabajan solamente con datos numéricos por lo que la codificación de campos es opcional según el modelo con el que se quiera trabajar, en este caso algunos modelos de clasificación no requieren dicha codificación.
    La columna fecha se estandarizará con un formato determinado (UTC) YYYY-MM-DDTHH:MM:SS.sssZ. Se cambian los nombres de los canales mal registrados, esto como consecuencia de un error que se detecto en el registro de actividades, específicamente en el campo 'canal'.

    \item \textbf{Filtrado de datos:} Se aplican filtros para seleccionar y extraer los datos relevantes para el proyecto, descartando aquellos que no cumplen con ciertos criterios o condiciones específicas. Se excluyen registros sin métodos asociados o con métodos inconsistentes. Además, se excluyen los registros que contienen el campo de RUT no anonimizado, cumpliendo con la política de privacidad de la empresa. Este enfoque permite centrarse en la información más relevante y útil.
\end{itemize}

Es importante destacar que la planificación de las transformaciones considera el orden y la secuencia adecuada de ejecución, así como la documentación de cada paso y los criterios de validación y verificación para garantizar la calidad de los datos transformados. Por lo tanto, se ha diseñado un plan detallado con las transformaciones antes mencionadas que se presentará en la siguiente sección.