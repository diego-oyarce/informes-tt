En esta etapa se determinan las fuentes de datos a ser usadas para el proyecto, incluyendo bases de datos, archivos .CSV y APIs. Esto además comprende la definición de la estructura, el formato y ubicación de cada fuente de datos dentro del proyecto.

La fuente de datos corresponde a un archivo .CSV que contiene información de la navegación web de los clientes en forma de Web Logs. Los Web Logs registrados vienen con 4 atributos, especificados a continuación:

\begin{itemize}
\item \textbf{rut cliente}: Este atributo representa un identificador único por cliente.
\item \textbf{fecha evento}: Representa la fecha y hora de la interacción del cliente con el sitio web.
\item \textbf{metodo}: Este atributo representa cuál fue el método al cual el cliente llamó al interactuar con el sitio web.
\item \textbf{canal}: Corresponde al canal web con el cual el cliente realizó la interacción en el sitio web.
\end{itemize}

Para almacenar la fuente de datos se utiliza una estructura de carpetas, siendo las siguientes:

\begin{itemize}
\item \textbf{Input}: Dentro de esta carpeta se encontrará el archivo .CSV tal cual es entregado.
\item \textbf{Intermediate}: Aquí se almacenará el archivo con la información preprocesada.
\item \textbf{Output}: Dentro de esta última carpeta se almacenará la información ya procesada y lista para ser usada.
\end{itemize}