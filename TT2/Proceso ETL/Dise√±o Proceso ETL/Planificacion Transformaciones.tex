Dentro de esta etapa, se lleva a cabo la planificación detallada de las transformaciones necesarias para construir una base sólida y consistente que respalde el desarrollo del proyecto. Estas transformaciones comprenden una serie de pasos destinados a limpiar, filtrar, combinar y enriquecer los datos de manera apropiada.

La planificación de las transformaciones desempeña un papel fundamental para asegurar la calidad y la integridad de los datos que se utilizarán en el proyecto. Durante esta fase, se identifican las tareas específicas que deben ejecutarse para lograr los objetivos establecidos, teniendo en cuenta los requisitos del proyecto y las necesidades del negocio.

Algunas de las transformaciones comunes incluyen \cite{etl-toolkit}:

\begin{itemize}
    \item \textbf{Limpieza de datos:} Se llevan a cabo tareas de limpieza para corregir errores, eliminar valores duplicados o inconsistentes y garantizar la coherencia de los datos. En este caso, la limpieza de datos se utilizará para estandarizar el formato y realizar la transformación y codificación de los campos. Inicialmente, se eliminarán todas las filas que contengan valores nulos. Luego, la columna 'fecha' se separará en: año, mes, día, hora, minuto y segundo, y posteriormente se eliminará la columna 'fecha'. Además, se codificarán los campos 'método' y 'canal' utilizando la 'codificación de etiqueta', como se señaló anteriormente. La lógica implementada en esta codificación asignará valores numéricos de forma ascendente a medida que aparezcan en el DataFrame, facilitando así la construcción de un modelo de aprendizaje automático.

    \item \textbf{Filtrado de datos:} Se aplican filtros para seleccionar y extraer los datos relevantes para el proyecto, descartando aquellos que no cumplen con ciertos criterios o condiciones específicas. Se excluyen registros sin métodos asociados o con métodos inconsistentes. Además, se excluyen los registros que contienen el campo de RUT no anonimizado, cumpliendo con la política de privacidad de la empresa. Este enfoque permite centrarse en la información más relevante y útil.
\end{itemize}

Es importante destacar que la planificación de las transformaciones considera el orden y la secuencia adecuada de ejecución, así como la documentación de cada paso y los criterios de validación y verificación para garantizar la calidad de los datos transformados. Por lo tanto, se ha diseñado un plan detallado con las transformaciones que se presentará en la siguiente sección.