El presente documento de Trabajo de Titulación tiene como objetivo mostrar la forma y el plan de trabajo que se utilizan a lo largo del proceso de desarrollo del proyecto propuesto. Además, se presentan los resultados del proceso investigativo que se ha realizado hasta la fecha, incluyendo estudios sobre el comportamiento de los clientes en canales web, modelos y algoritmos de predicción, y cómo desarrollar un proceso ETL y un Análisis Exploratorio de Datos (EDA).

El objetivo principal de este proyecto es analizar el comportamiento de los clientes de AFP Capital y sus preferencias de uso en un período de hasta 1 meses, con el fin de predecir futuras navegaciones personalizadas.

El proyecto consta de cuatro fases para su desarrollo. La primera fase abarca la planificación y el planteamiento de los antecedentes generales para la realización del proyecto. La segunda fase se centra en la investigación de la problemática en estudio, basándose en la situación actual planteada. La tercera fase abarca el modelamiento y desarrollo del proyecto, incluyendo el modelamiento de datos y el enfoque del proceso ETL y EDA. Esta fase también involucra el desarrollo del código que respaldará y ejecutará el modelo predictivo, mediante la construcción de bases de datos, APIs y la realización de pruebas para mitigar posibles errores encontrados. Por su parte el EDA contempla un análisis profundo de los datos entregados por la empresa. La cuarta y última fase concluye el desarrollo del proyecto y se enfoca en las conclusiones y recomendaciones, donde se presentarán las conclusiones obtenidas a lo largo del proceso y se elaborará un manual de usuario con las recomendaciones de uso.

Además, este proyecto se lleva a cabo bajo un marco de trabajo de desarrollo ágil, utilizando Scrum, y se estan usando metodologías de análisis y minería de datos, como CRISP-DM y OSEMN. El entorno de desarrollo se basa en Python, junto con bibliotecas de análisis y minería de datos como Pandas y Numpy, y frameworks de desarrollo de APIs como Flask, Django y FastAPI.


\textbf{Palabras clave:}  Afiliado, Administradora de Fondos de Pensiones, API (Application Programming Interfaces), EDA (Exploratory Data Analysis), Algoritmos de predicción, Algoritmos de clasificación, Modelos de predicción, ETL (Extract, Transform and Load), ARIMA (Modelo de Autorregresión integrada de media móvil), SARIMA (Modelo Estacional de Autorregresión integrada de media móvil), Redes Neuronales Artificiales (ANN), Redes LSTM (Long Short-Term Memory), Redes Neuronales Recurrentes (RNN).
