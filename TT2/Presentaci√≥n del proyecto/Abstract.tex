The purpose of this thesis is to show the form and the work plan used throughout the development process of the proposed project. In addition, the results of the research process that has been carried out to date are presented, including studies on customer behavior in web channels, predictive models and algorithms, and how to develop an ETL process.

The main objective of this project is to analyze AFP Capital's customer behavior and usage preferences over a period of up to 6 months, in order to predict future personalized browsing.

The project consists of four phases for its development. The first phase covers the planning and general background for the realization of the project. The second phase focuses on the investigation of the problem under study, based on the current situation. The third phase covers the modeling and development of the project, including the data modeling and ETL process approach. This phase also involves the development of the code that will support and execute the predictive model, through the construction of databases, APIs and testing to mitigate possible errors found. The fourth and last phase concludes the development of the project and focuses on conclusions and recommendations, where the conclusions obtained throughout the process will be presented, and a ...

\textbf{Keywords:} Affiliate, Pension Fund Administrator, API (Application Programming Interfaces), EDA (Exploratory Data Analysis), Predictive Algorithms, Classification Algorithms, Predictive Models, ETL (Extract, Transform and Load) ARIMA (AutoRegressive Integrated Moving Average), SARIMA (Seasonal AutoRegressive Integrated Moving Average), Artificial Neural Network (ANN), LSTM Networks (Long Short-Term Memory), Recurrent Neural Network (RNN).