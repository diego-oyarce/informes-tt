The purpose of this Degree Project document is to show the form and work plan used throughout the development process of the proposed project. In addition, the results of the research process that has been carried out to date are presented, including studies on customer behavior in web channels, predictive models and algorithms, and how to develop an ETL process and an Exploratory Data Analysis (EDA).

The main objective of this project is to analyze AFP Capital customers' behavior and usage preferences over a period of up to 1 month, in order to predict future personalized browsing.

The project consists of four phases for its development. The first phase covers the planning and general background for the implementation of the project. The second phase focuses on the investigation of the problem under study, based on the current situation. The third phase covers the modeling and development of the project, including data modeling and the ETL and EDA process approach. This phase also involves the development of the code that will support and execute the predictive model, through the construction of databases, APIs and testing to mitigate possible errors found. The EDA includes an in-depth analysis of the data provided by the company. The fourth and last phase concludes the development of the project and focuses on conclusions and recommendations, where the conclusions obtained throughout the process will be presented and a user manual with recommendations for use will be prepared.

In addition, this project is carried out under an agile development framework, using Scrum, and data mining and analysis methodologies, such as CRISP-DM and OSEMN, are being used. The development environment is based on Python, together with data mining and analysis libraries such as Pandas and Numpy, and API development frameworks such as Flask, Django and FastAPI.

\textbf{Keywords:} Affiliate, Pension Fund Administrator, API (Application Programming Interfaces), EDA (Exploratory Data Analysis), Predictive Algorithms, Classification Algorithms, Predictive Models, ETL (Extract, Transform and Load) ARIMA (AutoRegressive Integrated Moving Average), SARIMA (Seasonal AutoRegressive Integrated Moving Average), Artificial Neural Network (ANN), LSTM Networks (Long Short-Term Memory), Recurrent Neural Network (RNN).