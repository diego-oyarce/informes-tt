This thesis document comprehensively summarizes the complete process carried out to develop a machine learning model capable of predicting user behavior in AFP Capital's new virtual platform. In addition to detailing the creation of this model, it focuses on the relevance of predicting customer behavior and its strategic value for companies today, highlighting the importance of obtaining and managing accurate data on their behaviors.

The paper explores several learning models considered as viable options, delving into the detailed implementation of three specific models and presenting the results obtained. It highlights the process of selecting the most efficient model among those evaluated, describing in detail this chosen model and documenting the tests performed to demonstrate its final effectiveness.

In addition to the development of the model, the creation of a consumable service that generates predictions using the aforementioned model is addressed. The implementation of an API that facilitates the generation of predictions based on the input data required for the model is described.

The project concludes with the dockerization of all the work done, allowing its simplified deployment on the company's virtual platform. This approach seeks to ensure the accessibility and ease of deployment of the service.

Finally, recommendations for improvement and continuation of the project are included in case the company wishes to continue using this model. In addition, conclusions and reflections of the working group on the development and findings achieved during the project are presented.

\textbf{Key words:}  Affiliate, Pension Fund Administrator, API (Application Programming Interfaces), EDA (Exploratory Data Analysis), Prediction Algorithms, Classification Algorithms, Prediction Models, ETL (Extract, Transform and Load), ARIMA (Moving Average Integrated Autoregression Model), SARIMA (Seasonal Moving Average Integrated Autoregression Model), Artificial Neural Networks (ANN), LSTM (Long Short-Term Memory) Networks, Recurrent Neural Networks (RNN).