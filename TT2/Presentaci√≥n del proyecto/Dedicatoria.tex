Este trabajo se lo dedico principalmente a mi madre, Fabiola Trejo, quien ha sido un ejemplo de esfuerzo y lealtad. Sin ella, nada de lo que he logrado hasta ahora podría haber sido posible. Su fortaleza y compromiso me llevaron a donde estoy ahora. Estaré eternamente agradecido por su apoyo.

También quiero dar las gracias a mi hermano y a todas mis amistades que me han acompañado durante el proceso de terminar esta carrera. Su compañía fue uno de los pilares que me sostuvo todo este tiempo.

\textit{Diego Oyarce Trejo}


\vspace{2cm}

\newpage
Al reflexionar sobre el viaje que ha sido mi carrera universitaria, me siento profundamente agradecido por las numerosas personas que han iluminado mi camino, especialmente en mi inmersión en el fascinante mundo de la informática.

En primer lugar, quiero expresar mi más sincera gratitud a la Profesora Paula Castro Opazo. Su guía experta y la generosidad al compartir su conocimiento han sido fundamentales en el desarrollo y éxito de este proyecto. Su mentoría no sólo ha enriquecido mi experiencia académica, sino que también ha sembrado semillas de sabiduría que florecerán en mi futura carrera.

A mis amados padres, Pamela y Marcelo, les debo mi más profundo agradecimiento. Su amor incondicional, apoyo constante y sabias orientaciones han sido el faro que ha guiado mi viaje. Han sido mi fuente de fortaleza y motivación, permitiéndome encontrar y seguir mi verdadero camino.

A mi amada hermana Constanza, le agradezco por estar siempre presente, escuchando mis preocupaciones y ofreciendo consejos valiosos. Su apoyo ha sido un regalo inestimable en mi vida.

Quiero extender un especial agradecimiento a mis abuelos, Blanca y Rafael, cuya presencia durante mi infancia ha sido un regalo invaluable. Su amor inagotable, sus consejos sabios y su guía constante han sido faros de luz en mi formación. Su influencia ha dejado una huella indeleble en mi corazón y en mi vida, moldeando la persona que soy hoy. Por todo el amor que me han brindado y por sus enseñanzas, les estoy eternamente agradecido.

También quiero reconocer a mi amplia familia, cuyas palabras de aliento y apoyo han resonado fuertemente en cada paso que he dado.

A mis amigos, Andrés, Chelito, David, Cata, Luciano, y muchos otros, gracias por ser parte esencial de mi proceso académico y una invaluable fuente de alegría y camaradería. Los momentos que hemos compartido y el apoyo mutuo que nos hemos brindado ocuparán siempre un lugar especial en mi corazón.

Finalmente, este proyecto no solo marca el cierre de un capítulo significativo en mi vida, sino que también simboliza el comienzo de nuevos horizontes. A todas las personas mencionadas, y a muchas otras que han contribuido a mi viaje, les extiendo mi más sincero agradecimiento. Han dejado una huella imborrable en mi vida y siempre estarán en mi corazón.

\textit{Marcelo Tapia Riquelme}

\vspace{2cm}

\newpage
Realmente no sé cómo comenzar esto, solo sé que este es el final de una etapa en mi vida, la cual quizás duro un poco más de lo que hubiese querido, pero no por eso no estoy contento con todo lo que logre aprender, personas que conocí, amistades que se forjaron y experiencias obtenidas.

Cada término de un proceso marca el comienzo de uno nuevo y es por eso, que estoy más que agradecido con todas las personas que estuvieron conmigo durante este camino, las personas que están y que estarán en este nuevo comienzo. Podría nombrar a muchas personas con las cuales pude conectar y nutrirme para poder sobrepasar los obstáculos que se iban presentando en mi camino académico, pero cada uno de ellos sabe quienes son y si en su momento no les agradecí como era debido, les digo gracias, muchas gracias por haber compartido conmigo su tiempo, sobre todo sabiendo que lo material no es lo más importante, si no el tiempo por el cual nosotros como personas transitamos y lo que se hace con él.

Agradezco y dedico este trabajo a toda mi familia, la que se encuentra con nosotros y a mis abuelos Carlos González y Adriana Silva que en paz descansan, a pesar de todo nunca dejaron de apoyarme y ayudarme a alcanzar mis metas, no puedo no agradecer a mis abuelos Ricardo Gárate y Ana Vásquez, de los cuales aprendí que todo el esfuerzo y trabajo duro vale la pena, y por sobre todo que lo más importante siempre es la educación, es lo único que nosotros como personas podemos dejar a las generaciones venideras. Aun cuando yo les decía a mis padres que estaba cansado que me dijeran cada vez que los visitábamos “Estudie mijito, que es lo único que sus padres le pueden dejar”, reconozco que siempre tuvo la razón, gracias Tata.

\textit{Cristóbal González Gárate}