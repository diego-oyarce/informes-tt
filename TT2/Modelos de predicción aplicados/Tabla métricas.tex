\renewcommand{\tablename}{Tabla}


A continuación, se presenta una tabla que resume las distintas métricas ocupadas para conocer la precisión de los modelos aplicados.

Al tener 3 tipos de modelos distinos, las métricas varian entre cada uno de ellos, para el modelo arima las métricas representan el error que el modelo tiene al predecir una serie de tiempo, mientras más cercano a 0, mayor es su precisión y para el modelo secuencial las métricas representan la precisión de la predicción del modelo de clasificación.

\begin{center}
        \begin{tabular}{|c|c|c|c|c|c|c|}
            \multicolumn{7}{c}%
            {{\tablename\ Métricas.1 -- Tabla resumen métricas de métodos aplicados}} \\
            \hline
            \multirow{2}{*}{\textbf{Modelos aplicados}} &
                \multicolumn{6}{c|}{\textbf{Métricas}} \\
            \cline{2-7}
            & Precission & Recall & F1 Score & MAE & MSE & RMSE\\
            \hline
            Modelo ARIMA & - & - & - & 17.87 & 372.87 & 19.31 \\
            \hline
            Modelo Secuencial & 57.94\% & 54.87\% & 53.66\% & - & -& - \\
            \hline
        \end{tabular}
\end{center}

A raíz de nuestra decisión de no continuar con el desarrollo del modelo de autoencoders, nos encontramos en la situación de no contar con métricas de desempeño para este modelo en nuestro proyecto. Esta decisión se basa en el reconocimiento de que el autoencoder, aunque eficaz en la identificación de patrones y anomalías dentro de datos de alta dimensión, no se alinea con nuestra necesidad de predecir comportamientos y tendencias futuras. Dado que las métricas de desempeño para autoencoders están diseñadas principalmente para evaluar la precisión en la reconstrucción de datos y no para medir la efectividad en la predicción de eventos futuros, estas métricas resultan irrelevantes para nuestros objetivos. Por ende, al descartar el uso del autoencoder, también descartamos la aplicación de sus métricas asociadas, orientándonos ahora hacia la búsqueda de otros modelos predictivos que posean métricas de desempeño alineadas con nuestra meta de anticipar y actuar sobre futuros comportamientos y tendencias.