\documentclass[letterpaper, 12pt]{report}
\usepackage[utf8]{inputenc}
\usepackage[spanish]{babel}
\usepackage{graphicx}
\usepackage[left=4cm, right=2.5cm, top=4cm, bottom=2.5cm]{geometry}
\usepackage{titlesec}
\usepackage{caption}
\usepackage{verbatim}
\usepackage{tocloft}
\usepackage{float}
\usepackage[hidelinks]{hyperref}
\usepackage{apacite}
\usepackage{comment}
\usepackage{amsmath}
\usepackage{parskip}
\usepackage{setspace}
\usepackage{listings}
\usepackage{fancyhdr}


\onehalfspacing %Espacio de 1,5 líneas
% Título, autores y fecha
%\title{\LARGE \begin{doublespace} MODELO DE APRENDIZAJE AUTOMÁTICO PARA LA PREDICCIÓN DEL COMPORTAMIENTO DE CLIENTES EN CANAL WEB \end{doublespace} \\ \vspace{1cm}
%TRABAJO DE TÍTULACIÓN PARA OPTAR AL TÍTULO DE INGENIERO CIVIL EN COMPUTACIÓN MENCIÓN INFORMÁTICA}
%\author{Diego Oyarce Trejo \\ doyarce@utem.cl \and
%        Marcelo Ignacio Tapia Riquelme \\ marcelo.tapiar@utem.cl \and
%        Cristóbal Andres González Gárate \\ cristobal.gonzalezg@utem.cl \\
%        \vspace{1.0cm}
%        \\Profesora Guía: Paula Castro Opazo \\Ingeniera Civil en Computación Mención en Informática \\Magister en Data Science \\paula.castro@utem.cl
%        }
%\date{\today}
\renewcommand{\headrulewidth}{0pt}
\pagestyle{fancy}

\renewcommand{\contentsname}{Índice}

\begin{document}

\begin{titlepage}
  \centering
  \begin{tabular}{@{}p{2cm}@{\hspace{1cm}}p{12cm}@{}}
    \begin{picture}(0,0)
      \put(0,-70){\includegraphics[width=2cm,height=3cm]{img/logocolor.png}}
    \end{picture}
    &
    \raggedright
    \normalsize UNIVERSIDAD TECNOLÓGICA METROPOLITANA\\
    \normalsize FACULTAD DE INGENIERÍA\\
    \normalsize DEPARTAMENTO DE INFORMÁTICA Y COMPUTACIÓN\\
    \normalsize ESCUELA DE INFORMÁTICA
  \end{tabular}

  \vspace{2.5cm}
  \LARGE \begin{doublespace} MODELO DE APRENDIZAJE AUTOMÁTICO PARA LA PREDICCIÓN DEL COMPORTAMIENTO DE CLIENTES EN CANAL WEB \end{doublespace}
  \vspace{0.1cm}
  \large \begin{singlespace}
    TRABAJO DE TÍTULACIÓN PARA OPTAR AL TÍTULO DE INGENIERO CIVIL EN COMPUTACIÓN MENCIÓN INFORMÁTICA
  \end{singlespace}

  \raggedleft
  \vspace{0.5cm}
  \textbf{\normalsize AUTORES:} \\
  \vspace{0.1cm}
  \normalsize GONZÁLEZ GÁRATE, CRISTÓBAL ANDRES \\
  \vspace{0.1cm}
  \normalsize TAPIA RIQUELME, MARCELO IGNACIO \\
  \vspace{0.1cm}
  \normalsize OYARCE TREJO, DIEGO ESTEBAN \\
  \vspace{0.5cm}
  \textbf{\normalsize PROFESORA GUÍA:} \\
  \normalsize CASTRO OPAZO, PAULA

  \vspace{0.5cm}
  \centering
  \normalsize SANTIAGO - CHILE \\
  \normalsize 2023


\end{titlepage}

\fancyhf{}

\pagenumbering{roman}
\renewcommand{\thepage}{\roman{page}}

\tableofcontents

\begin{singlespace}
  \listoffigures
\end{singlespace}


\setcounter{section}{1}

\chapter{Presentación del proyecto}

\section{Resumen}
El presente documento de Trabajo de Titulación tiene como objetivo mostrar la forma y el plan de trabajo que se utilizan a lo largo del proceso de desarrollo del proyecto propuesto. Además, se presentan los resultados del proceso investigativo que se ha realizado hasta la fecha, incluyendo estudios sobre el comportamiento de los clientes en canales web, modelos y algoritmos de predicción, y cómo desarrollar un proceso ETL y un Análisis Exploratorio de Datos (EDA).

El objetivo principal de este proyecto es analizar el comportamiento de los clientes de AFP Capital y sus preferencias de uso en un período de hasta 1 meses, con el fin de predecir futuras navegaciones personalizadas.

El proyecto consta de cuatro fases para su desarrollo. La primera fase abarca la planificación y el planteamiento de los antecedentes generales para la realización del proyecto. La segunda fase se centra en la investigación de la problemática en estudio, basándose en la situación actual planteada. La tercera fase abarca el modelamiento y desarrollo del proyecto, incluyendo el modelamiento de datos y el enfoque del proceso ETL y EDA. Esta fase también involucra el desarrollo del código que respaldará y ejecutará el modelo predictivo, mediante la construcción de bases de datos, APIs y la realización de pruebas para mitigar posibles errores encontrados. Por su parte el EDA contempla un análisis profundo de los datos entregados por la empresa. La cuarta y última fase concluye el desarrollo del proyecto y se enfoca en las conclusiones y recomendaciones, donde se presentarán las conclusiones obtenidas a lo largo del proceso y se elaborará un manual de usuario con las recomendaciones de uso.

Además, este proyecto se lleva a cabo bajo un marco de trabajo de desarrollo ágil, utilizando Scrum, y se estan usando metodologías de análisis y minería de datos, como CRISP-DM y OSEMN. El entorno de desarrollo se basa en Python, junto con bibliotecas de análisis y minería de datos como Pandas y Numpy, y frameworks de desarrollo de APIs como Flask, Django y FastAPI.


\textbf{Palabras clave:}  Afiliado, Administradora de Fondos de Pensiones, API (Application Programming Interfaces), EDA (Exploratory Data Analysis), Algoritmos de predicción, Algoritmos de clasificación, Modelos de predicción, ETL (Extract, Transform and Load), ARIMA (Modelo de Autorregresión integrada de media móvil), SARIMA (Modelo Estacional de Autorregresión integrada de media móvil), Redes Neuronales Artificiales (ANN), Redes LSTM (Long Short-Term Memory), Redes Neuronales Recurrentes (RNN).


\section{Abstract}
%The purpose of this thesis is to show the form and the work plan used throughout the development process of the proposed project. In addition, the results of the research process that has been carried out to date are presented, including studies on customer behavior in web channels, predictive models and algorithms, and how to develop an ETL process.

The main objective of this project is to analyze AFP Capital's customer behavior and usage preferences over a period of up to 6 months, in order to predict future personalized browsing.

The project consists of four phases for its development. The first phase covers the planning and general background for the realization of the project. The second phase focuses on the investigation of the problem under study, based on the current situation. The third phase covers the modeling and development of the project, including the data modeling and ETL process approach. This phase also involves the development of the code that will support and execute the predictive model, through the construction of databases, APIs and testing to mitigate possible errors found. The fourth and last phase concludes the development of the project and focuses on conclusions and recommendations, where the conclusions obtained throughout the process will be presented, and a ...

\textbf{Keywords:} Affiliate, Pension Fund Administrator, API (Application Programming Interfaces), EDA (Exploratory Data Analysis), Predictive Algorithms, Classification Algorithms, Predictive Models, ETL (Extract, Transform and Load).

\clearpage
\pagenumbering{arabic}
\renewcommand{\thepage}{\arabic{page}}

\section{Palabras Clave}
\begin{singlespace}
    \begin{itemize}
        \item Afiliado
        \item Administradora de Fondos de Pensiones
        \item API (Application Programming Interfaces)
        \item EDA (Exploratory Data Analysis)
        \item Algoritmos de predicción
        \item Algoritmos de clasificación
        \item Modelos de predicción
        \item ETL (Extract, Transform and Load)
    \end{itemize}
\end{singlespace}

\section{Descripción del trabajo de título}
El trabajo de titulación se basa en un proyecto empresarial que tiene como objetivo procesar los registros de navegación del sitio web para afiliados de AFP Capital. El propósito principal es detectar comportamientos de los clientes y sus preferencias de uso, con el fin de personalizar las futuras experiencias de navegación. La lectura de los registros se realizará extrayendo la información desde Kibana, una plataforma basada en ElasticSearch, que registra la información a través de diversas APIs utilizadas en el sitio web.

Los elementos fundamentales del proyecto incluyen el análisis exploratorio de datos, extracciones, transformaciones, cargas, modelos de predicción y detección de preferencias. El objetivo final es desarrollar un modelo capaz de predecir el comportamiento de los clientes en el canal web.

\section{Objetivos}
\subsubsection{Objetivo general}
Analizar el comportamiento de los clientes y sus preferencias de uso en un período igual o inferior a 6 meses, para predecir navegaciones futuras personalizadas. 

\subsubsection{Objetivos específicos }
\begin{itemize}
    \item Realizar una investigación de las herramientas utilizadas para la predicción de comportamiento de usuarios en un canal web.
    \item Llevar a cabo un análisis y estudio de los datos entregados por la empresa. 
    \item Realizar un proceso ETL con la información de navegación web de los clientes de AFP Capital, para analizar su comportamiento dentro del sitio web privado. 
    \item Desarrollar un modelo capaz de predecir el comportamiento de los clientes de AFP Capital, para entregar navegaciones personalizadas futuras.
    \item Establecer recomendaciones de personalización en función de los hallazgos del modelo de predicción para futuras navegaciones dentro del sitio web de AFP Capital.
\end{itemize}

\section{Alcances y Limitaciones}
\subsubsection{Alcances}

El proyecto contempla los siguientes alcances:

\begin{itemize}
\item Se analizará el comportamiento de los clientes de AFP Capital en su nuevo sitio web privado.
\item El proyecto entregará un modelo capaz de predecir el comportamiento de los clientes de AFP Capital en el sitio web, así como una API que permita obtener recomendaciones de comportamiento personalizadas para un afiliado específico.
\end{itemize}

\subsubsection{Limitaciones}

El proyecto tiene las siguientes limitaciones:

\begin{itemize}
\item No se contará con acceso directo a las bases de datos de AFP Capital, por lo tanto, se trabajará con una muestra de datos.
\item No se podrá acceder a información sensible de los clientes de AFP Capital, como los RUTs (Rol Único Tributario) u otra información personal identificable.
\item El análisis se basará únicamente en datos cualitativos de la navegación web de los usuarios.
\item La disponibilidad de datos se limita a un periodo de 15 días debido a las dificultades asociadas con su extracción.
\end{itemize}

\chapter{La empresa}

\section{Historia}
La historia de AFP Capital se remonta a noviembre de 1980, cuando se implementó en Chile el sistema de pensiones de capitalización individual. El 16 de enero de 1981, se constituyó la sociedad Administradora de Fondos de Pensiones Santa María, que más tarde se transformaría en AFP Capital S.A. Desde sus inicios, la empresa se destacó por su filosofía de servicio, enfocada en satisfacer las necesidades y expectativas de sus afiliados. 
En 1995, AFP Capital estableció la filial Santa María Internacional S.A., con el propósito de expandir su alcance y ofrecer servicios a personas naturales o jurídicas del extranjero, así como invertir en AFP o sociedades relacionadas con materias previsionales en otros países. Esta iniciativa consolidó la presencia de AFP Capital en el ámbito internacional y fortaleció su posición como una administradora de fondos de pensiones líder en la región. 
En el año 2000, se produjo una relevante transacción en la historia de AFP Capital. ING Group adquirió Aetna Inc., incluyendo el 96,56\% de las acciones de AFP Capital S.A. Esta adquisición tuvo como objetivo reforzar la posición de liderazgo de AFP Capital en el mercado previsional chileno y contribuir a su crecimiento y desarrollo. 
Posteriormente, en 2008, AFP Capital llevó a cabo una fusión con AFP Bansander, otra reconocida administradora de fondos de pensiones en Chile. Esta fusión permitió consolidar aún más las operaciones de AFP Capital y fortalecer su presencia en el país. A fines de 2011, Grupo SURA, una empresa líder en el negocio de pensiones en Latinoamérica, adquirió las operaciones de ING en la región. Esta adquisición llevó a AFP Capital a formar parte de Grupo SURA y a beneficiarse de su amplia experiencia y recursos, consolidándose como una compañía destacada en el mercado previsional latinoamericano. En resumen, la historia de AFP Capital está marcada por su constante evolución, consolidación y liderazgo en el mercado de administración de fondos de pensiones en Chile. A lo largo de los años, ha demostrado su compromiso con la excelencia en la prestación de servicios previsionales y su capacidad de adaptación a los cambios y desafíos del entorno económico y regulatorio. 

%agregar la imágen de la historia%

\section{Descripción general}
AFP Capital es una destacada compañía chilena dedicada al negocio de pensiones y administración de fondos de pensiones. Forma parte de SURA, una reconocida empresa que ofrece servicios financieros y previsionales en Chile y otros países de América Latina. El enfoque principal de AFP Capital es proporcionar a sus afiliados asesoría personalizada y servicios diferenciados que les permitan alcanzar una mejor pensión al momento de su jubilación. La empresa se distingue por su compromiso con la optimización de la calidad de sus servicios, la entrega de información transparente y relevante a sus afiliados, y su solidez empresarial.

Con más de tres décadas de experiencia en el mercado, AFP Capital se ha posicionado como una de las principales administradoras de fondos de pensiones en Chile. Esto se debe en gran medida a su administración seria, responsable y eficiente en el manejo de los Fondos de Pensiones, así como a su enfoque prudente y estratégico en la inversión y gestión de los recursos.

La compañía cuenta con un equipo de colaboradores altamente capacitados y comprometidos, quienes contribuyen a la excelencia en la atención al cliente y al logro de los objetivos financieros de los afiliados. Además, AFP Capital se destaca por su constante innovación y adaptación a los cambios regulatorios y a las necesidades cambiantes de los afiliados, con el fin de ofrecer soluciones efectivas y satisfactorias en el ámbito de las pensiones.


\section{Misión y visión}
\subsubsection{Misión}
La misión de AFP Capital es: "Acompañamos a nuestros clientes, a través de una asesoría experta y diferenciadora en soluciones de ahorro para alcanzar su número, su Pensión, creciendo sustentablemente, desarrollando a nuestros colaboradores e integrándose responsablemente a la comunidad." \cite{afpcapital}

\subsubsection{Visión}
La visión de AFP Capital es: "Somos Guías, acompañamos a nuestros clientes a lograr sus sueños a través del ahorro." \cite{afpcapital}

\chapter{Marco teórico}

\section{Importancia de predecir el comportamiento del cliente en un sitio web}
La predicción del comportamiento del cliente dentro de un entorno web se considera a la aplicación de técnicas y modelos analíticos para lograr predecir en cierta manera las posibles necesidades, acciones, preferencias y decisiones que un cliente pueda tomar mientras interactúa en alguna plataforma en línea o sitio web. En los últimos años, ha sido de gran importancia la predicción del comportamiento de los clientes para las empresas, gracias a esto buscan anticipar las necesidades y preferencias de sus clientes, pudiendo adaptar los productos y servicios para entregar una mayor satisfacción al cliente (Zheng, Thompson, Lam, Yoon y Gnanasambandam, 2013). La lealtad de los clientes representa un valor clave para las empresas, ya que un cliente leal seguirá consumiendo los productos y servicios de la empresa, por lo que si se mejora la experiencia del usuario, la satisfacción del cliente aumenta y esto genera un aumento en la ganancia de la empresa. 
Según Zheng, Thompson, Lam, Yoon y Gnanasambandam (2013), la predicción del comportamiento del cliente ayuda a las empresas a identificar oportunidades de mejora y mercado, además de ayudar a tomar decisiones informadas sobre estrategias de publicidad y marketing. El objetivo fundamental de predecir el comportamiento del cliente en un entorno web es lograr comprender y anticipar las acciones de los clientes con la meta de personalizar, mejorar la experiencia de usuario y poder aumentar la satisfacción y fidelidad de los clientes. 
Las predicciones pueden abarcar distintos aspectos del comportamiento de un cliente dentro de un canal web, a grandes rasgos existen 4 tipos de predicciones que se pueden realizar, están las predicciones de compras, donde mediante el análisis de patrones de navegación, su historial de compras, preferencias y características demográficas, gracias a esto se busca predecir las compras futuras de un cliente, se encuentra la predicción de clics, esta busca anticipar los enlaces o elementos con los cuales un cliente va a interactuar dentro de un sitio web, lo que busca mejorar la calidad de contenido que se encuentra desplegado y lograr mejorar la usabilidad del sitio web, también está presente la predicción de abandono de carrito, esta permite tomar acciones de recuperación o retención del cliente, se concentra en identificar aquellos clientes que agregan productos a un carrito de compra pero no finalizan el proceso de compra y por ultimo, esta la predicción de retención de clientes, esta busca predecir qué clientes están más cercanos a abandonar o terminar la relación existente con el sitio web, para poder generar e implementar estrategias para aumentar la fidelización y retención de estos clientes. 



\section{Comportamiento del cliente/afiliado en el canal web}
\subsection{Definición y relevancia del comportamiento del cliente para el negocio}
Considerando los modelos de negocio establecidos por las Administradoras de Fondos de Pensiones (AFP), surge la importancia de la figura del cliente. Según la Real Academia Española, un cliente es una persona que realiza una compra o utiliza los servicios ofrecidos por un profesional o empresa (Real Academia Española, s.f). Sin embargo, en el contexto de las AFP, los clientes se denominan afiliados, ya que contribuyen o están inscritos en un plan de pensiones \cite{pension-system}.

El afiliado es el centro del negocio y su importancia radica principalmente en la rentabilidad que aporta. Cada trabajador que decide afiliarse representa una ganancia, mientras que cada afiliado que decide desafiliarse genera una pérdida. Además, la experiencia del servicio que brinda la AFP hacia el afiliado es crucial, ya que puede promover la marca si es positiva. En tercer lugar, el afiliado, al ser una fuente de ganancias para el modelo, puede contribuir al crecimiento de la empresa al darle su preferencia. Además, la experiencia del cliente y su retroalimentación son valiosas, ya que pueden proporcionar conocimientos sobre los puntos débiles y las áreas de mejora del sistema \cite{def-cliente}.

Dentro de las diferentes funciones que tiene el cliente, en primer lugar, se encuentra el cliente como consumidor. Esta es una de las funcionalidades más tradicionales, ya que el objetivo intrínseco del cliente es consumir o contratar servicios. Como consumidor, adquiere un producto o servicio y lo utiliza para satisfacer una necesidad, lo que representa la principal fuente de ingresos para la empresa.

En segundo lugar, se encuentra el cliente como "prosumidor", es decir, alguien que consume y produce al mismo tiempo \cite{understand-roles}. Además de consumir, el cliente también deja reseñas o realiza comentarios en lugares especializados, lo cual es útil para generar información que mejore la experiencia del servicio.

En tercer lugar, se considera al cliente como crítico, ya que si la experiencia del cliente es negativa, los comentarios y reseñas negativas que proporcione pueden tener un impacto constructivo o destructivo.

En cuarto lugar, el cliente es una pieza fundamental en el desarrollo de productos y servicios. Los comentarios de los clientes pueden guiar el desarrollo de servicios innovadores que se ajusten a las necesidades que ellos indican. En el caso específico de las AFP, esto se refiere a los afiliados.

En quinto lugar, el cliente se desempeña como evaluador de la experiencia. Relacionado con los puntos anteriores, la mejor manera de mejorar la experiencia del cliente es tener en cuenta sus comentarios sobre este aspecto, lo que puede marcar la diferencia frente a otras empresas competidoras en el mercado.

Por último, el cliente puede convertirse en un embajador eventual de la marca, es decir, puede promover el negocio mediante recomendaciones, comentarios y reseñas positivas.

\subsection{Características del comportamiento del cliente en el canal web}
Para comprender la experiencia y el comportamiento del cliente dentro de un canal web, es importante reconocer la existencia del consumer journey, el cual describe las distintas etapas por las que un cliente pasa al momento de consumo de un producto o servicio. Según Lemon y Verhoef (2016) las etapas corresponden a conciencia, investigación, consideración, compra, uso y evaluación. La definición de conciencia da cuenta de la necesidad o el problema que debe ser resuelto, mientras que investigación refiere de la búsqueda de información por parte del cliente para posibles soluciones, comparando entre las distintas opciones disponibles (Lemon y Verhoef, 2016). Luego la etapa de consideración donde el cliente puede evaluar entre las opciones disponibles escogiendo la que mejor se adapta a sus necesidades dando paso a la etapa de compra cuando el cliente contrata y/o compra el mejor servicio a su parecer. Posterior viene la etapa de uso donde el cliente puede experimentar y testear la calidad, funcionalidad y experiencia del servicio dando pie a la última etapa que consiste en evaluar la experiencia como satisfactoria o insatisfactoria con la entrega voluntaria de feedback tanto positivo como negativo. Por lo tanto las posibles opciones disponibles para los clientes dentro del canal web buscan hacer del consumer journey una eficiente y grata experiencia. 
Para poder acceder al canal web de AFP Capital, se debe estar afiliado y tener una cuenta privada personal [Rut y Contraseña] y una vez se hace ingreso al canal web privado, el afiliado tiene disponibles variadas opciones para realizar y que buscan satisfacer sus posibles necesidades, estas corresponden al pago o no de la cotización mensual, la obtención de certificados de cotizaciones, certificado de afiliación, certificado de antecedentes previsionales, certificados de traspaso de fondos, certificado de vacaciones progresivas y certificados tributarios, como también la obtención de certificados generales, como el certificado de residencia, certificado de suscripción de ahorro previsional voluntario [APV], certificado de cuenta 2, certificado de remuneraciones imponibles, certificado de periodos no cotizados y certificado de trabajo pesado, si el afiliado es una persona pensionada puede obtener certificado de asignacion familiar, certificado de calidad pensionado, certificado de pensiones pagadas, certificado de pensión en trámite, certificado de ingreso base y certificado de comprobante de pago de pensión, también poder hacer obtención de la cartola en línea. El canal web privado permite realizar el ahorro obligatorio y ahorrar voluntariamente, dentro de una cuenta de ahorro previsional voluntario [APV] o cuenta 2, realizar inversiones, hacer depósitos directos, tener planillas de pagos y ver las comisión cobrada como afiliado. También le otorga al afiliado la opción de ver su fondo de pensiones, ver los tipos de fondo de pensión, tipo A, tipo B, tipo C, tipo D, tipo E y sus porcentajes de rentabilidad, realizar un cambio de fondo de pensiones y recibir educación previsional. Le otorga al afiliado la opción de realizar giros en sus cuentas personales, acceder a rescates financieros y realizar el trámite de pensión. 



\subsection{Factores que afectan el comportamiento del cliente}
% Factores que influyen en el comportamiento del cliente en el canal web, tales como la usabilidad y el diseño del sitio web.
\cite{lemon2016customer} proponen que los principales factores que influyen en el comportamiento del usuario y su experiencia son los sensoriales, afectivos, cognitivos, puntos de contacto y externos. La experiencia sensorial se refiere a los aspectos perceptibles por los sentidos del cuerpo, como la vista, el olfato y el tacto. En cuanto a la experiencia afectiva, se debe considerar la emocionalidad del cliente como resultado de la experiencia con el producto o servicio. En el aspecto cognitivo, se refiere a los pensamientos, creencias y actitudes que el cliente puede tener hacia la compañía, el producto o el servicio \cite{lemon2016customer}. Los puntos de contacto hacen referencia a las diferentes formas en que el cliente y la compañía interactúan, como la publicidad, el servicio al cliente, las redes sociales o las interacciones transaccionales. Por último, el factor externo se refiere al contexto actual, las condiciones socioeconómicas y otros factores que pueden afectar la experiencia del usuario y que están fuera del control de la compañía.

Dentro de los factores que pueden influir en el comportamiento de un cliente en el canal web, se encuentran principalmente la usabilidad y el diseño. En cuanto a la usabilidad, depende de siete características que garantizan una buena experiencia para el usuario. Según Sánchez \cite{Usabilidad=softw}, la accesibilidad, legibilidad, navegabilidad, facilidad de aprendizaje, velocidad de utilización, eficiencia del usuario y tasas de error del canal web influyen en la experiencia del usuario y en el feedback que este pueda brindar sobre el uso de los servicios.

Por otro lado, el diseño del sitio web depende de cinco características para garantizar un buen contenido y estética, y lograr que el usuario encuentre lo que busca en el menor tiempo posible, es decir, eficiencia. El autor Walter Sánchez \cite{Usabilidad=softw} indica que el diseño debe ser entendible, novedoso, comprensible, inteligente y atractivo, lo que permite acercar los contenidos de mejor manera al usuario y lograr una navegación más intuitiva. Estos factores son de gran importancia para que el usuario pueda encontrar el contenido que busca en el menor tiempo posible y tener una experiencia positiva al interactuar con la interfaz del sitio web.


\section{Herramientas para la predicción del comportamiento del cliente en el canal web}

\subsection{Introducción a las herramientas de análisis de datos}
En el entorno empresarial actual, la capacidad de tomar decisiones informadas y basadas en datos se ha vuelto fundamental para el éxito y la competitividad de las organizaciones. El análisis de datos desempeña un papel crucial en este proceso, permitiendo a las empresas obtener información valiosa a partir de grandes volúmenes de datos y utilizarla para comprender el comportamiento del cliente de manera más profunda y precisa. Esto resulta de suma importancia, ya que la calidad de las decisiones tomadas marca la diferencia entre el éxito y el fracaso \cite{analitica-predictiva}.

Dentro de las herramientas de análisis de datos, se destacan cuatro conceptos clave que han revolucionado la forma en que se procesan y se obtiene información de los datos: Business Intelligence, Big Data, Machine Learning y Data Mining. Estas herramientas proporcionan a las empresas la capacidad de extraer conocimientos y patrones significativos de los datos, lo que a su vez les permite tomar decisiones estratégicas más acertadas y personalizar sus estrategias de marketing y atención al cliente.

El Business Intelligence (BI) se refiere a la recopilación, análisis y presentación de datos empresariales para facilitar la toma de decisiones. Mediante el uso de diversas técnicas y herramientas, el BI permite a las empresas visualizar y comprender mejor los datos de sus operaciones y clientes. Esto incluye la generación de informes, el análisis de tendencias, la monitorización de indicadores clave de rendimiento (KPI) y la creación de tableros de control interactivos. El BI ayuda a las organizaciones a identificar oportunidades, detectar áreas de mejora y optimizar su rendimiento en función de datos históricos y en tiempo real. Sobre la inteligencia de negocios, se ha determinado que cada implementación es única para cada proceso empresarial \cite{analitica-empresarial}.

El Big Data se refiere a la gestión y análisis de grandes volúmenes de datos, tanto estructurados como no estructurados, que superan la capacidad de las herramientas tradicionales de almacenamiento y procesamiento. El Big Data se caracteriza por las tres V's: Volumen (gran cantidad de datos), Velocidad (alta velocidad de generación y procesamiento de datos) y Variedad (diversidad de fuentes y formatos de datos). Para aprovechar el potencial del Big Data, las empresas emplean técnicas de procesamiento distribuido y herramientas específicas para el almacenamiento, procesamiento y análisis de estos datos masivos. El análisis de Big Data permite identificar patrones, tendencias y correlaciones ocultas en los datos, lo que brinda información valiosa para entender y anticipar el comportamiento del cliente.

El Machine Learning (aprendizaje automático) es una rama de la inteligencia artificial que permite a los sistemas informáticos aprender y mejorar automáticamente a partir de la experiencia sin ser programados explícitamente. En lugar de basarse en una analítica descriptiva, el Machine Learning ofrece una analítica predictiva \cite{analitica-empresarial}. Mediante algoritmos y modelos, el Machine Learning permite a las empresas analizar grandes conjuntos de datos y detectar patrones complejos en el comportamiento del cliente. Esto permite realizar predicciones y recomendaciones personalizadas, así como automatizar tareas y procesos, lo que mejora la eficiencia operativa y la experiencia del cliente.

El Data Mining (minería de datos) se refiere al proceso de descubrir información valiosa, patrones y relaciones desconocidas en grandes conjuntos de datos. Utilizando técnicas estadísticas y algoritmos avanzados, el Data Mining permite identificar correlaciones y tendencias ocultas en los datos, lo que ayuda a las empresas a comprender mejor el comportamiento del cliente y tomar decisiones más acertadas. Esta herramienta es especialmente útil para la segmentación de clientes, la detección de fraudes, la recomendación de productos y la personalización de ofertas.


\subsection{Métodos, técnicas y tecnologías de análisis de datos}
%Métodos, técnicas y tecnologías de análisis de datos utilizados en la predicción del comportamiento del cliente
En la actualidad, el análisis de datos desempeña un papel fundamental en la predicción del comportamiento del cliente. Las empresas y organizaciones buscan comprender y anticiparse a las necesidades y preferencias de sus clientes para mejorar la toma de decisiones y ofrecer productos y servicios más personalizados. Para lograr esto, se han desarrollado diversos métodos, técnicas y tecnologías que permiten analizar grandes volúmenes de datos y extraer información valiosa. 
A continuación, se listan algunos de los métodos, técnicas y tecnologías más utilizados en el análisis de datos para predecir el comportamiento del cliente.
\vspace{0.5cm}

\textbf{\large Métodos y modelos}
\begin{itemize}
    \item Regresión logística
    \item Clustering
    \item Árboles de decisión
    \item Random Forest
    \item Gradient Boosting Machine
\end{itemize}
\vspace{0.5cm}

\textbf{\large Técnicas}
\begin{itemize}
    \item Redes neuronales artificiales (ANN)
    \item Support Vector Machine (SVM)
\end{itemize}
\vspace{0.5cm}

\textbf{\large Tecnologías} 
\begin{itemize}
    \item Tableau
    \item Python (con bibliotecas como Pandas, NumPy, Scikit-learn)
    \item R (con paquetes como dplyr, caret, randomForest)
    \item Apache Spark
    \item KNIME
    \item RapidMiner
    \item QlikView
    \item Power BI
\end{itemize}


\subsection{Modelos de predicción de comportamiento del cliente}
En la era digital, los modelos de predicción de comportamiento del cliente son un recurso fundamental para las empresas que buscan tomar decisiones informadas y personalizar sus estrategias. Este subcapítulo del informe se adentrará en los diversos modelos utilizados para anticipar las acciones y preferencias de los clientes, destacando su relevancia en la toma de decisiones estratégicas y la mejora de la experiencia del usuario en el entorno online.

\begin{itemize}
    \item  \subsubsection{Modelos de regresión}
    \noindent
    La regresión logística corresponde a un algoritmo de aprendizaje automático supervisado que es empleado para resolver problemas de clasificación. Si bien, su nombre contiene “regresión”, en realidad corresponde a un método de clasificación.

    Se da uso a la regresión logística cuando la variable de respuesta o variable objetivo es categórica. En lugar de predecir un valor numérico como en la regresión lineal, la regresión logística estima la probabilidad de que una observación pertenezca a una categoría específica.

    Los modelos de regresión logística se basan en la función logística, también conocida como función sigmoide, que mapea cualquier valor real a un rango entre 0 y 1. La función sigmoide tiene la siguiente forma matemática:

    \begin{equation*}
        f(z) = \frac{1}{(1 + e^{-z})}
    \end{equation*}

    En la regresión logística, se ajusta un modelo lineal a los datos de entrada y se aplica la función sigmoide al resultado para obtener la probabilidad de pertenencia a una clase. La ecuación del modelo se expresa como:

    \begin{equation*}
        p(y=1|x) = \frac{1}{(1 + e^{(-(b0 + b1x1 + b2x2 + ... + bn*xn))})}
    \end{equation*}

    Donde:

    p(y=1|x) es la probabilidad condicional de que la variable de respuesta sea igual a 1 dada la entrada x.

    b0, b1, b2, ..., bn son los coeficientes del modelo que se ajustan durante el proceso de entrenamiento.

    x1, x2, ..., xn son los valores de las variables de entrada.

    El proceso de ajuste de la regresión logística implica encontrar los mejores valores para los coeficientes del modelo con la finalidad de maximizar la verosimilitud de los datos observados. Esto se puede hacer mediante métodos numéricos como la maximización de la función de verosimilitud o mediante algoritmos de optimización como el gradiente descendente.

    Una vez entrenado el modelo, se puede utilizar para hacer predicciones clasificando nuevas observaciones según la probabilidad estimada. Por ejemplo, si la probabilidad estimada de pertenencia a una clase es superior a un umbral (generalmente 0.5), se clasificará como perteneciente a esa clase.

    Para nuestro caso en particular, puede ser utilizado el modelo de regresión logística para predecir el comportamiento de usuarios en un canal web, para ello se necesitaría tener datos históricos que contengan información relevante sobre el comportamiento pasado de los usuarios y las variables predictoras asociadas. Estas variables predictoras pueden incluir características demográficas, patrones de uso del sitio web o aplicación, historial de compras, interacciones anteriores, entre otros.

    Una vez que se tienen los datos y las variables predictoras, se puede entrenar un modelo de regresión logística utilizando técnicas de ajuste como la maximización de la verosimilitud o el gradiente descendente. Una vez entrenado el modelo, puede ser utilizado para predecir el comportamiento futuro de los usuarios en función de nuevas observaciones o datos entrantes.

    Es importante tener en consideración que la calidad de las predicciones dependerá de la calidad de los datos utilizados para entrenar el modelo y de la selección adecuada de las variables predictoras. Además, es fundamental realizar una validación adecuada del modelo utilizando técnicas como la validación cruzada o la separación de conjuntos de entrenamiento y prueba para evaluar su rendimiento y generalización en datos no vistos.
    
    \begin{itemize}
        \item \textbf{Ventajas de los modelos de regresión logística}
            \begin{itemize}
                \item \textbf{Interpretación de resultados:} La regresión logística proporciona coeficientes que indican la dirección y la magnitud de la relación entre las variables predictoras y la variable de respuesta. Esto permite interpretar el efecto relativo de cada variable en la probabilidad de pertenecer a una clase específica.
                \item \textbf{Manejo de variables independientes categóricas:} La regresión logística puede manejar tanto variables independientes continuas como categóricas. Incluso puede manejar variables categóricas con más de dos categorías mediante técnicas como la codificación de variables ficticias.
                \item \textbf{Estimación de probabilidades:} La regresión logística estima la probabilidad de pertenencia a una clase específica en lugar de simplemente clasificar observaciones en categorías. Esto es útil cuando se necesita una medida de certeza o riesgo asociado con la clasificación.
                \item \textbf{Buena capacidad de generalización:} La regresión logística puede funcionar bien con conjuntos de datos pequeños o moderados, y es menos propensa al sobreajuste en comparación con otros algoritmos más complejos. Esto la hace adecuada para aplicaciones con muestras limitadas.   
            \end{itemize}
        \item \textbf{Desventajas de los modelos de regresión logística}
            \begin{itemize}
                \item \textbf{Linealidad de la relación:} La regresión logística asume una relación lineal entre las variables predictoras y la probabilidad logarítmica de la variable de respuesta. Si existe una relación no lineal, la regresión logística puede no ajustarse adecuadamente o requerir transformaciones adicionales de las variables.
                \item \textbf{Sensible a valores atípicos y datos faltantes:} Los valores atípicos o datos faltantes pueden afectar negativamente el rendimiento de la regresión logística. Es necesario manejarlos adecuadamente para evitar sesgos o imprecisiones en los resultados.
                \item \textbf{Suposición de independencia:} La regresión logística asume que las observaciones son independientes entre sí. Si hay dependencias o correlaciones entre las observaciones, la precisión de los resultados puede verse comprometida.
                \item \textbf{No apto para problemas no lineales:} Si existe una relación compleja y no lineal entre las variables predictoras y la variable de respuesta, la regresión logística puede no ser el modelo más adecuado. En tales casos, se pueden requerir técnicas más avanzadas, como modelos no lineales o de aprendizaje profundo.
            \end{itemize}
    \end{itemize}

\end{itemize}

\\
Los modelos de recomendación son algoritmos y técnicas utilizados en sistemas de recomendación para ofrecer sugerencias personalizadas a los usuarios. Estos modelos se utilizan en una amplia gama de aplicaciones, como plataformas de comercio electrónico, servicios de streaming de música y video, redes sociales y más \cite{crear-recomendacion}.

El objetivo de un modelo de recomendación es predecir o sugerir elementos que sean relevantes o interesantes para un usuario en particular, basándose en su historial de preferencias, comportamiento pasado o en información de usuarios similares. Estos modelos aprovechan el poder del aprendizaje automático y la minería de datos para analizar patrones y relaciones en grandes conjuntos de datos.

Existen varios tipos de modelos de recomendación, entre los más comunes se encuentran \cite{recomendation-systems}:

\begin{itemize}
    \item Filtrado colaborativo: Este enfoque se basa en la idea de que si a un grupo de usuarios con preferencias similares les gusta un conjunto de elementos, entonces a un usuario nuevo con características similares también le podrían gustar esos elementos. El filtrado colaborativo utiliza la información de las interacciones pasadas de los usuarios (por ejemplo, clasificaciones o historial de compras) para generar recomendaciones.
    \item Filtrado basado en contenido: Este enfoque utiliza información sobre las características y atributos de los elementos para recomendar otros elementos similares. Por ejemplo, en un servicio de streaming de música, se pueden recomendar canciones o artistas similares a los que un usuario ha escuchado anteriormente en función de género, estilo o letras.
    \item Modelos híbridos: Estos modelos combinan múltiples enfoques, como filtrado colaborativo y basado en contenido, para aprovechar sus fortalezas y proporcionar recomendaciones más precisas y personalizadas.
\end{itemize}

Los modelos de recomendación se construyen utilizando técnicas de aprendizaje automático, como regresión logística, árboles de decisión, redes neuronales o algoritmos de factorización matricial. Estos modelos se entrenan utilizando conjuntos de datos históricos que contienen información sobre las preferencias y elecciones de los usuarios, y luego se aplican en tiempo real para generar recomendaciones en función de nuevos datos.

\begin{itemize}
    \item \textbf{Ventajas de los modelos de recomendación}
    \begin{itemize}
        \item \textbf{Personalización:} Los modelos de recomendación ofrecen sugerencias personalizadas a los usuarios, lo que mejora la experiencia del usuario y facilita la búsqueda de productos o contenido relevante.
        \item \textbf{Descubrimiento de nuevos elementos:} Los modelos de recomendación pueden ayudar a los usuarios a descubrir nuevos elementos que podrían ser de su interés, ampliando así sus opciones y experiencias.
        \item \textbf{Mejora de la retención y fidelidad de los usuarios:} Al proporcionar recomendaciones precisas y relevantes, los modelos de recomendación pueden aumentar la satisfacción del usuario, mejorar la retención y fomentar la fidelidad a la plataforma o servicio.
        \item \textbf{Eficiencia en la toma de decisiones:} Los usuarios pueden ahorrar tiempo y esfuerzo al recibir sugerencias personalizadas, lo que les ayuda a tomar decisiones más rápidas y eficientes.
    \end{itemize}
    \item \textbf{Desventajas de los modelos de recomendación}
    \begin{itemize}
        \item \textbf{Sesgo y burbujas de filtro:} Los modelos de recomendación pueden verse afectados por el sesgo inherente en los datos de entrenamiento y pueden crear burbujas de filtro, limitando la diversidad y la exposición a nuevas ideas o perspectivas.
        \item \textbf{Fracaso en captar preferencias cambiantes:} Los modelos de recomendación pueden tener dificultades para captar las preferencias cambiantes de los usuarios a medida que sus gustos y necesidades evolucionan con el tiempo.
        \item \textbf{Problemas de inicio en frío:} Los modelos de recomendación pueden tener dificultades para ofrecer recomendaciones precisas para nuevos usuarios o elementos que tienen una falta de información histórica.
        \item \textbf{Privacidad y preocupaciones éticas:} Los modelos de recomendación recopilan y utilizan datos de los usuarios, lo que puede plantear preocupaciones de privacidad y cuestiones éticas relacionadas con el manejo de la información personal.
    \end{itemize}
\end{itemize}

\subsubsection{Modelos de series temporales}
Los modelos de series temporales son técnicas utilizadas para analizar y predecir datos secuenciales que están organizados en función del tiempo. En una serie temporal, los datos se registran en intervalos regulares (como horas, días, meses, etc.) y cada punto de datos está asociado con una marca de tiempo.

El objetivo principal de los modelos de series temporales es comprender y capturar los patrones, tendencias y estacionalidad en los datos a lo largo del tiempo, y utilizar esta información para hacer predicciones futuras. Estos modelos son ampliamente utilizados en diversos campos, como la economía, las finanzas, la meteorología, la demanda de productos, la planificación de inventario y más.

Los modelos de series temporales se basan en la suposición de que los datos pasados pueden proporcionar información útil para predecir el futuro. Algunos de los modelos más comunes utilizados en el análisis de series temporales son:
\begin{itemize}
    \item \textbf{Media móvil (MA):} Este modelo estima el valor futuro de la serie temporal en función de un promedio de los errores pasados. Se utiliza para capturar patrones aleatorios o no sistemáticos en los datos.
    \item \textbf{Autoregresión (AR):} Este modelo estima el valor futuro de la serie temporal en función de valores pasados de la propia serie. Se utiliza para capturar la dependencia de la serie en sí misma a lo largo del tiempo.
    \item \textbf{Autoregresión de media móvil (ARMA):} Este modelo combina los enfoques AR y MA para capturar tanto la dependencia de la serie en sí misma como los patrones aleatorios.
    \item \textbf{Autoregresión integrada de media móvil (ARIMA):} Este modelo amplía el modelo ARMA al considerar también las diferencias entre los valores de la serie temporal. Se utiliza para capturar tendencias y estacionalidad en los datos.
\end{itemize}

Además de estos modelos clásicos, también se utilizan enfoques más avanzados, como los modelos de espacio de estados, los modelos de suavizado exponencial y los modelos de redes neuronales recurrentes (RNN), que pueden capturar relaciones más complejas y no lineales en los datos de series temporales.

Es importante destacar que el análisis de series temporales requiere un enfoque cuidadoso para la selección del modelo, la identificación de patrones y la evaluación de la precisión de las predicciones. Además, se deben tener en cuenta factores como la estacionalidad, la estacionariedad de la serie y la presencia de datos faltantes o valores atípicos para obtener resultados confiables.

\textbf{Ventajas de los modelos de series temporales}

\begin{itemize}
    \item \textbf{Captura de patrones temporales:} Los modelos de series temporales pueden capturar patrones, tendencias y estacionalidad en los datos a lo largo del tiempo. Esto permite comprender mejor la dinámica de los datos y hacer predicciones más precisas.
    \item \textbf{Predicciones a corto plazo:} Los modelos de series temporales son adecuados para hacer predicciones a corto plazo, ya que utilizan la información histórica para predecir los valores futuros. Esto es especialmente útil en aplicaciones donde se necesita anticipar eventos próximos, como demanda de productos o pronóstico del clima.
    \item \textbf{Utilización de datos secuenciales:} Los modelos de series temporales aprovechan la estructura secuencial de los datos y utilizan la información de los puntos anteriores para hacer predicciones en el siguiente punto. Esto permite tener en cuenta la dependencia temporal en los datos y obtener resultados más precisos.
    \item \textbf{Flexibilidad en la elección del modelo:} Existen diferentes tipos de modelos de series temporales que se pueden utilizar según la naturaleza de los datos y los patrones presentes. Esto proporciona flexibilidad para seleccionar el modelo más adecuado para el problema específico.
\end{itemize}

\textbf{Desventajas de los modelos de series temporales}

\begin{itemize}
    \item \textbf{Sensibilidad a datos faltantes o valores atípicos:} Los modelos de series temporales pueden verse afectados negativamente por la presencia de datos faltantes o valores atípicos. Estos pueden distorsionar los patrones y afectar la precisión de las predicciones.
    \item \textbf{Dificultad con tendencias no lineales:} Los modelos de series temporales asumen a menudo que las relaciones son lineales o pueden ser capturadas por modelos lineales. Si hay tendencias no lineales en los datos, los modelos lineales pueden no ajustarse adecuadamente y se pueden requerir enfoques más avanzados.
    \item \textbf{Necesidad de datos históricos adecuados:} Los modelos de series temporales requieren una cantidad suficiente de datos históricos para hacer predicciones precisas. En ausencia de datos suficientes, los modelos pueden tener dificultades para capturar patrones y generar resultados confiables.
    \item \textbf{Problemas con cambios estructurales:} Si hay cambios estructurales significativos en los datos de series temporales (por ejemplo, cambios en la estacionalidad o en los patrones), los modelos de series temporales pueden tener dificultades para adaptarse y pueden requerir ajustes manuales.
\end{itemize}
\subsubsection{Modelos de atribución}

Los modelos de atribución permiten predecir el recorrido que los clientes seguirán al momento de concretar una compra. Este recorrido puede contener las redes sociales, el uso del sitio web del vendedor, el correo electrónico, entre otros. Los modelos de atribución permiten determinar el impacto que tiene el uso de las acciones para el sistema de marketing. Este tipo de modelo permite darle mayor importancia a los canales de marketing y a los puntos de contacto que existen entre el cliente y el vendedor, que llevaron al cliente a realizar una compra.

Al asignar crédito a sus canales de marketing y puntos de contacto, se puede aumentar la posibilidad de que los clientes logren concretar una compra, esto a través de la identificación de las áreas del recorrido del comprador que se puedan mejorar, la determinación del retorno de la inversión para cada canal o punto de contacto, el descubrimiento de las áreas más efectivas para gastar el presupuesto de marketing y la adaptación de las campañas de marketing y muestra de contenido totalmente personalizado por clientes \cite{modelo-atribucion}

Existen variados tipos de modelos de atribución, todos tienen el mismo procedimiento de asignar crédito a los canales y punto de contacto, cada uno de estos tipos de modelo le atribuyen un peso distinto a cada canal y punto de contacto. Los modelos a continuación son los más aptos para lograr la predicción del comportamiento de un cliente:

\begin{itemize}
\item \textbf{Modelo de atribución Multi-Touch:} Este modelo demuestra ser poderoso ya que tiene en cuenta todos los canales y puntos de contacto con los que los clientes interactúan a lo largo de su camino al concretar una compra. Deja en evidencia cuáles de los canales y punto de contacto fueron más influyentes y de cómo estas trabajaron en conjunto para influenciar al cliente.
\item \textbf{Modelo de atribución Lineal:} Corresponde a un tipo de modelo de atribución Multi-Touch que le entrega el mismo peso a cada uno de los canales y puntos de contacto con los que el cliente interactúa en su camino al concretar una compra.
\item \textbf{Modelo de atribución Time-Decay:} También llamado modelo de atribución de declive en el tiempo, además de considerar todos los puntos de contacto, también considera el tiempo que cada uno de estos puntos de contacto ocurrió, por lo que, los puntos de contacto o interacciones que sucedieron más cercano al momento en que se concretó la compra reciben mayor peso.
\end{itemize}

\textbf{Ventajas de los modelos de atribución}

\begin{itemize}
\item \textbf{Facilita el rastrear de mejor manera el paso a paso del cliente:} Esto gracias a la atención que se le entrega a cada canal y punto de contacto con el cual el cliente interactúa a la hora de concretar una compra.
\item \textbf{Permiten mayor personalización de rastreo de los clientes:} Al saber que canales y punto de contacto tiene cada uno de los clientes, se puede llegar a entregar una experiencia personalizada a cada uno de los clientes.
\item \textbf{Comprender la contribución de cada canal y punto de contacto:} Permite comprender como cada canal y punto de contacto contribuye a lograr los objetivos comerciales. Siendo de gran ayuda para identificar como asignar los recursos de manera mas efectiva y lograr optimizar las estrategias.
\item \textbf{Identificar canales y puntos de contacto de alto rendimiento:} Un modelo de atribución puede revelar qué canales o puntos de contacto tienen un mayor interacción con los clientes en términos de generación de resultados. Esto permite a las empresas enfocar sus recursos en los canales más efectivos y maximizar su retorno de inversión.
\end{itemize}

\textbf{Desventajas de los modelos de atribución}

\begin{itemize}
\item \textbf{Poseen una mayor complejidad que los otros modelos:} La implementación de un modelo de atribución puede ser compleja y requerir un enfoque personalizado según las necesidades y características de cada empresa. Además, no hay un modelo de atribución único que sea universalmente aceptado, lo que puede generar falta de consenso y confusión en la industria.
\item \textbf{La interpretación de los resultados puede ser subjetiva:} La interpretación de los resultados de un modelo de atribución puede estar sujeta a la interpretación y suposiciones del analista. Diferentes personas pueden llegar a conclusiones diferentes basadas en los mismos resultados, lo que puede generar cierta subjetividad en la interpretación de los datos.
\item \textbf{Poseen limitaciones en la medición del seguimiento:} El modelo de atribución depende de la disponibilidad y calidad de los datos. Si los datos son limitados o imprecisos, los resultados del modelo pueden no ser confiables o representativos de la realidad.
\end{itemize}
Los árboles de decisión son modelos de aprendizaje supervisado que se utilizan para predecir a qué clase o categoría pertenece un caso conocido mediante uno o más atributos. Estos modelos se construyen utilizando un algoritmo llamado \emph{partición binaria recursiva}. Durante el entrenamiento, el algoritmo realiza divisiones en un subconjunto de los datos basadas en decisiones asociadas a variables conocidas, generando así dos nuevos subconjuntos. Este proceso se repite de manera recursiva hasta alcanzar un punto de terminación predefinido, lo que resulta en la creación del clasificador basado en árbol de decisión. Luego, cada nuevo dato, que posee atributos conocidos, sigue las ramificaciones del árbol siguiendo las reglas y decisiones generadas durante el proceso de entrenamiento.

En la actualidad, los árboles de decisión son unos de los modelos de aprendizaje más utilizados debido a su buen rendimiento \cite{arboles-decision}. Estos algoritmos pueden generar modelos predictivos tanto para variables cuantitativas (regresión) como para variables cualitativas o categóricas (clasificación).

Como se mencionó anteriormente, un árbol de decisión realiza tareas de clasificación. Un clasificador es un algoritmo que nos permite asignar sistemáticamente una clase a cada uno de los casos presentados.

\begin{figure}[H]
    \begin{minipage}[t]{0.9\textwidth}
        \caption{Estructura de un árbol de decisión}
        \label{arbol-de-decision}        
    \end{minipage}

    \vspace{10pt}

    \begin{minipage}[b]{1.1\textwidth}
        \centering
        \includegraphics[width=\textwidth]{img/estructura-arbol-de-decision.jpg}        
    \end{minipage}

    \begin{minipage}[t]{0.9\textwidth}
        Fuente: Aprende IA. Recuperado de \url{https://aprendeia.com/arboles-de-decision-clasificacion-teoria-machine-learning/}
    \end{minipage}
\end{figure}

En la figura anterior se puede visualizar la estructura que posee un árbol de decisión, en este se aprecia como actua el algoritmo de partición binaria mencionado al comienzo, tomando un conjunto y separandolo en subconjuntos hasta llegar a un final previamente establecido.

Para estimar la precisión de un clasificador, se calcula la tasa de error de clasificación verdadera. Esta tasa se obtiene evaluando un conjunto de valores X a los que el clasificador asigna una clase incorrecta, y se divide por el total de valores en X. Idealmente, se debería conocer la clase de todos los casos en el universo antes del entrenamiento, o en su defecto, de una muestra de tamaño similar al universo. Sin embargo, en la mayoría de los casos reales, no se dispone de todos los datos del universo, por lo que se trabaja con una muestra y se estima la tasa de error mencionada anteriormente utilizando \emph{estimadores internos}.

\subsubsection{Ventajas de los árboles de decisión}
\begin{itemize}
    \item \textbf{Interpretabilidad:} Los árboles de decisión son fácilmente interpretables y comprensibles para los humanos. La estructura del árbol se puede visualizar de manera intuitiva, lo que permite comprender cómo se toman las decisiones y qué atributos son más relevantes para la clasificación.
    \item \textbf{Facilidad de uso:} La construcción y el uso de un árbol de decisión son relativamente sencillos en comparación con otros algoritmos de aprendizaje automático más complejos. No requieren una preparación exhaustiva de los datos ni un procesamiento previo complicado. Además, los árboles de decisión pueden manejar datos numéricos y categóricos sin requerir transformaciones adicionales, lo que simplifica el flujo de trabajo de modelado.
    \item \textbf{Capacidad para manejar datos faltantes y variables irrelevantes:} Los árboles de decisión tienen la capacidad de manejar datos faltantes en los atributos de forma natural. Durante la construcción del árbol, si un atributo tiene valores faltantes, el modelo puede utilizar otros atributos para tomar decisiones sin requerir imputación de datos. Además, los árboles de decisión son resistentes a variables irrelevantes, lo que significa que pueden ignorar atributos que no aportan información útil para la clasificación.
    \item \textbf{Flexibilidad y robustez:} Los árboles de decisión pueden manejar tanto problemas de clasificación como de regresión. Además, son capaces de capturar relaciones no lineales entre los atributos y la variable objetivo. Aunque cada árbol individual puede ser susceptible al sobreajuste, se pueden aplicar técnicas de regularización, como la poda, para mejorar la generalización y evitar el sobreajuste.
    \item \textbf{Eficiencia en tiempo de entrenamiento y predicción:} Los árboles de decisión tienen tiempos de entrenamiento y predicción rápidos, ya que solo implican la evaluación de una serie de reglas de decisión. Aunque el tiempo de construcción puede ser mayor para conjuntos de datos grandes, una vez construido, el árbol puede ser utilizado eficientemente para hacer predicciones en tiempo real.
\end{itemize}

\subsubsection{Desventajas de los árboles de decisión}
\begin{itemize}
    \item \textbf{Sensibilidad a cambios pequeños en los datos:} Los árboles de decisión son muy sensibles a cambios pequeños en los datos de entrenamiento. Una modificación mínima en los datos de entrada puede dar lugar a un árbol de decisión completamente diferente. Esto puede hacer que el modelo sea inestable y su rendimiento pueda variar significativamente.
    \item \textbf{Tendencia al sobreajuste:} Los árboles de decisión tienen la capacidad de adaptarse demasiado a los datos de entrenamiento. Si no se controla adecuadamente, el árbol puede memorizar el ruido o las fluctuaciones aleatorias en los datos de entrenamiento, lo que puede resultar en un mal rendimiento en datos nuevos y no vistos. La poda y otras técnicas de regularización se utilizan para mitigar este problema.
    \item \textbf{Limitaciones en la representación de relaciones complejas:} Aunque los árboles de decisión pueden capturar relaciones no lineales entre atributos y la variable objetivo, pueden tener dificultades para representar relaciones complejas que requieren una combinación de múltiples atributos. Las decisiones tomadas en cada nodo se basan en un solo atributo, lo que puede limitar su capacidad para modelar interacciones más sofisticadas.
    \item \textbf{Propensión a sesgos en los datos de entrenamiento:} Los árboles de decisión pueden verse afectados por sesgos en los datos de entrenamiento, especialmente cuando hay desequilibrios en las clases o falta representación de ciertas categorías. Esto puede resultar en una clasificación desigual o inexacta en casos minoritarios o poco representados.
\end{itemize}
El algoritmo random forest corresponde a un algoritmo empleado en machine learning registrado por Leo Breiman y Adele Cutler \cite{random-forest}, este combina la salida de múltiples árboles de decisión para llegar a un resultado. El uso de random forest se ha hecho popular a causa de su facilidad de uso y flexibilidad, ya que puede ser empleado para problemas de clasificación y regresión.

El random forest se encuentra formado por varios árboles de decisión, los cuales son propensos a tener problemas como sesgos o sobreajuste, pero cuando se trata con una gran cantidad de árboles se logra llegar a resultados más precisos. ''Mientras que los árboles de decisión consideran todas las posibles divisiones de características, los bosques aleatorios solo seleccionan un subconjunto de esas características.'' \cite{random-forest}

Random forest cuenta con tres hiperparámetros principales que se deben de configurar antes de iniciar el entrenamiento \cite{random-forest}:

\begin{itemize}
    \item Tamaño del nodo.
    \item Cantidad de árboles de decisión.
    \item Cantidad de características muestreadas. 
\end{itemize}

El algoritmo se encuentra compuesto de un conjunto de árboles de decisión, cada árbol del conjunto se encuentra compuesto de una muestra de datos, la cual proviene de un conjunto de entrenamiento con reemplazo, llamada muestra de arranque \cite{random-forest}.

A partir de la muestra de entrenamiento, se extrae un porcentaje para reservarlos como datos de prueba, los cuales se conocen como muestra fuera de la bolsa (oob). Luego, se inyecta otra instancia de aleatoriedad mediante el agrupamiento de características, lo que agrega más diversidad al conjunto de datos y reduce la correlación entre los árboles de decisión \cite{random-forest}.

En un Random Forest, el proceso de predicción puede variar según el tipo de problema que se esté abordando \cite{random-forest}. En el caso de tareas de regresión, se utiliza un enfoque de promediado, donde las predicciones de los árboles de decisión individuales se promedian para obtener el valor final de la predicción. Esto proporciona una estimación más precisa y estable del resultado deseado. Por otro lado, en tareas de clasificación, se utiliza un enfoque de votación mayoritaria. Cada árbol de decisión emite su propia predicción y la clase que obtiene la mayoría de votos se selecciona como la clase predicha. Esto permite tomar una decisión conjunta basada en las opiniones de múltiples árboles, lo que puede mejorar la precisión en la clasificación de las muestras.

Finalmente, la muestra extraída en un comienzo, la muestra fuera de la bolsa (oob) será utilizada para realizar una validación cruzada, finalizando la predicción.

\begin{figure}[H]
    \begin{minipage}[t]{0.9\textwidth}
        \caption{Estructura de un random forest}
        \label{random-forest}        
    \end{minipage}

    \vspace{10pt}

    \begin{minipage}[b]{1.1\textwidth}
        \centering
        \includegraphics[width=\textwidth]{img/estructura-random-forest.png}        
    \end{minipage}

    \begin{minipage}[t]{0.9\textwidth}
        Fuente: IBM. Recuperado de \url{https://www.ibm.com/mx-es/topics/random-forest}
    \end{minipage}
\end{figure}

\subsubsection{Ventajas de random forest}
\begin{itemize}
    \item \textbf{Riesgo reducido de sobreajuste:} Los árboles de decisión corren el riesgo de sobre ajustarse, ya que tienden a ajustar todas las muestras que se encuentran dentro de los datos de entrenamiento. Sin embargo, cuando hay una gran cantidad de árboles de decisión dentro del random forest, el clasificador no será capaz de ajustarse demasiado al modelo, ya que el promedio de los árboles no correlacionados logra reducir la varianza general y el error de predicción.
    \item \textbf{Aporta Flexibilidad:} Debido a su capacidad para abordar con gran precisión tanto tareas de regresión como de clasificación, el método conocido como random forest es ampliamente utilizado por los científicos de datos. Además, su capacidad de agrupar características lo convierte en una herramienta eficaz para estimar valores faltantes, manteniendo la precisión incluso cuando falta parte de los datos.
    \item \textbf{Importancia de la característica fácil de determinar:} El random forest ofrece una forma conveniente de evaluar la importancia o contribución de las variables en un modelo. Existen varias formas de medir la importancia de las características. Por lo general, se utilizan el índice de Gini y la disminución media de impurezas (MDI) para evaluar cuánto afecta la exclusión de una variable específica a la precisión del modelo.
    Sin embargo, otra medida de importancia es la importancia de permutación, también conocida como precisión de disminución media (MDA). La MDA determina la disminución promedio en la precisión al permutar de forma aleatoria los valores de las características en las muestras out-of-bag (muestras que no se utilizan en el proceso de entrenamiento).
\end{itemize}

\subsubsection{Desventajas de random forest}
\begin{itemize}
    \item \textbf{Proceso que requiere mucho tiempo:} Debido a que los algoritmos de random forest son capaces de manejar conjuntos de datos extensos, suelen ofrecer predicciones más precisas. Sin embargo, es importante tener en cuenta que el procesamiento de datos puede volverse lento, ya que se deben calcular los datos para cada árbol de decisión de forma individual.  
    \item \textbf{Requiere más recursos:} Debido a que los random forest procesan conjuntos de datos más grandes, es cierto que se requieren más recursos para almacenar dichos datos. El aumento en el tamaño del conjunto de datos implica una mayor necesidad de memoria y capacidad de almacenamiento para garantizar un funcionamiento eficiente del algoritmo.    
    \item \textbf{Más Complejo:} La interpretación de la predicción de un solo árbol de decisiones resulta más sencilla en comparación con la interpretación de un conjunto de árboles de decisión.
\end{itemize}


\subsection{Tabla comparativa de los modelos de predicción}
\begin{table}[ht]
\captionsetup{font=small} % Ajusta el tamaño de fuente de la leyenda de la tabla
\small % Ajusta el tamaño de fuente de la tabla
\begin{tabular}{|p{0.2\linewidth}|p{0.27\linewidth}|p{0.27\linewidth}|p{0.26\linewidth}|}
\hline
\textbf{Modelo} & \textbf{Ventajas} & \textbf{Desventajas} & \textbf{Aplicaciones} \\
\hline
Modelos de Regresión & 
Proporciona una relación cuantitativa entre variables independientes y la variable de respuesta. & 
Supone una relación lineal entre variables, lo que puede no ser válido en todos los casos. & 
Predicción de valores numéricos continuos. \\
\hline
Modelos de Recomendación & 
Personalización de sugerencias para los usuarios. & 
Puede requerir una gran cantidad de datos y tener problemas con datos faltantes o sesgos inherentes. & 
Recomendaciones de productos en comercio electrónico. \\
\hline
Modelos de Series Temporales & 
Captura patrones temporales y estacionales en los datos a lo largo del tiempo. & 
Sensibilidad a valores atípicos y datos faltantes, y dificultad para capturar tendencias no lineales. & 
Predicción de la demanda de productos. Pronóstico del clima. \\
\hline
Modelos de Atribución & 
Permite cuantificar la contribución relativa de diferentes variables a un resultado o impacto. & 
Puede ser difícil determinar la verdadera relación causal entre las variables. & 
Evaluación del retorno de inversión (ROI) de una campaña publicitaria. \\
\hline
Modelos de Árboles de Decisión & 
Proporciona una estructura de decisiones fácilmente interpretable. & 
Pueden ser propensos al sobreajuste si no se controla adecuadamente. & 
Clasificación y predicción en diversos campos, como medicina, marketing y finanzas. \\
\hline
\end{tabular}
\end{table}


\subsection{Metodología del proyecto}
Para llevar a cabo el desarrollo del proyecto, se definieron cuatro fases que corresponden a la totalidad del proyecto, las cuales corresponden a:

\subsubsection{Fase 1: Planteamiento y planificación}

Para la primera fase del proyecto, se llevará a cabo una planificación de la manera en la que será abordada la problemática, para desarrollar un anteproyecto que será utilizado para evaluar y planificar las actividades correspondientes al desarrollo del proyecto. Entre ellas se encuentran:

\begin{itemize}
    \item Planteamiento del proyecto y sus objetivos.
    \item Definición de alcances y limitaciones.
    \item Creación de un cronograma de actividades.
\end{itemize}

\subsubsection{Fase 2: Investigación}

Para la segunda fase, se realizará una investigación de herramientas y recursos necesarios para llevar a cabo un diseño de la solución para la problemática del proyecto planteado, sumado a un análisis de las bases de datos brindadas por la empresa AFP Capital. Una vez realizado lo anterior, se llevará a cabo una propuesta de diseño para la problemática, siendo entregada y analizada por la empresa, con la finalidad de pasar a desarrollo. Algunas de las actividades de esta fase corresponden a:

\begin{itemize}
    \item Investigación del problema.
    \item Toma de requerimientos.
    \item Investigación de tecnologías de análisis de datos.
\end{itemize}

\subsubsection{Fase 3: Modelamiento y desarrollo}
Para la tercera fase, se llevará a cabo el diseño y desarrollo del sistema propuesto, además de realizar pruebas para verificar el correcto funcionamiento. Algunas de las actividades de esta fase corresponden a:

\begin{itemize}
    \item Modelado del sistema ETL.
    \item Modelado de la API.
    \item Implementación del modelo propuesto.
    \item Pruebas y validaciones.
    \item Correcciones de errores.
\end{itemize}

\subsubsection{Fase 4: Conclusiones y recomendaciones}
Para la última fase, se dará fin al desarrollo del proyecto, elaborando un manual de usuario el cual indicaría algunas funcionalidades del sistema. Algunas de las actividades de esta fase corresponden a:

\begin{itemize}
    \item Desarrollo de manual de usuario.
    \item Redacción de conclusiones y recomendaciones.
    \item Cierre del proyecto.
\end{itemize}


\subsection{Metodología del sistema}
\subsubsection{CRISP-DM}
La metodología CRISP-DM (Cross-Industry Standard Process for Data Mining) es un proceso estándar utilizado para realizar proyectos de minería de datos. La metodología CRISP-DM se divide en seis fases distintas que se describen a continuación:

\begin{enumerate}
    \item \textbf{Comprensión del problema:} En esta fase se define el problema a resolver y se establecen los objetivos del proyecto. También se recopilan los datos necesarios para el proyecto.
    \item \textbf{Comprensión de los datos:} En esta fase se realiza una exploración de los datos para comprender su calidad, estructura y relevancia para el problema en cuestión.
    \item \textbf{Preparación de los datos:} En esta fase se limpian y procesan los datos para que puedan ser utilizados en la etapa de modelado.
    \item \textbf{Modelado:} En esta fase se aplican técnicas de modelado para desarrollar un modelo predictivo. Se prueban diferentes modelos y se selecciona el que mejor se ajuste a los datos.
    \item \textbf{Evaluación:} En esta fase se evalúa el modelo desarrollado en la fase anterior. Se verifica que el modelo funcione correctamente y se ajuste adecuadamente a los datos.
    \item \textbf{Implementación:} En esta fase se implementa el modelo desarrollado en la fase de modelado en un entorno de producción. También se establecen planes para monitorear el rendimiento del modelo y actualizarlo según sea necesario.
\end{enumerate}

Las fases de la metodología CRISP-DM son iterativas, lo que significa que es posible volver a una fase anterior si es necesario.

\subsubsection{OSEMN}

La metodología OSEMN (acrónimo de las palabras en inglés: Obtain, Scrub, Explore, Model, Interpret) es un proceso utilizado en la minería de datos y el análisis de datos para trabajar con grandes conjuntos de datos de manera efectiva. 

\begin{enumerate}
    \item \textbf{Obtener (Obtain):} En esta etapa, se recopilan los datos necesarios para el análisis. Los datos pueden provenir de diferentes fuentes, como bases de datos, archivos en línea o registros de sensores. La calidad y la cantidad de los datos obtenidos son cruciales para el éxito del análisis.
    \item \textbf{Limpieza (Scrub):} Una vez que se han obtenido los datos, es necesario realizar una limpieza para eliminar datos innecesarios o incorrectos. Esta etapa puede implicar la eliminación de duplicados, la corrección de errores y la eliminación de valores atípicos. El objetivo de esta etapa es obtener datos limpios y coherentes para el análisis.
    \item \textbf{Exploración (Explore):} En esta etapa, se utilizan técnicas de visualización y estadísticas para explorar los datos y obtener información sobre ellos. Se pueden identificar patrones, tendencias y relaciones entre diferentes variables. El objetivo es obtener una comprensión más profunda de los datos y de cómo se relacionan entre sí.
    \item \textbf{Modelado (Model):} En esta etapa, se utilizan técnicas de modelado estadístico o de aprendizaje automático para crear modelos que puedan predecir resultados futuros o identificar patrones en los datos. El objetivo es utilizar los datos para crear un modelo que pueda utilizarse para tomar decisiones informadas.
    \item \textbf{Interpretación (Interpret):} En esta etapa, se interpretan los resultados obtenidos en la etapa de modelado. Los resultados pueden ser utilizados para tomar decisiones o para generar nuevas hipótesis que puedan ser exploradas en futuros análisis.
\end{enumerate}

Se propone el uso de la metodología OSEMN, ya que se enfoca en el análisis de datos y la creación de modelos predictivos. OSEMN también es una metodología más flexible que CRISP-DM, lo que puede ser útil en un proyecto de SCRUM donde se busca una mayor adaptabilidad.

Por otro lado, también se propone el uso de la metodología CRISP-DM, ya que el proyecto incluye una etapa de exploración y análisis de datos, seguida por una fase de construcción de modelos. CRISP-DM se enfoca en el proceso completo de minería de datos, desde la comprensión del problema hasta la implementación del modelo, lo que puede servir para realizar un trabajo más estructurado.

Ya que este proyecto se encuentra bajo el marco de trabajo SCRUM, ambas metodologías pueden ser utilizadas de manera complementaria, utilizando OSEMN para las fases de creación de modelos y CRISP-DM para la etapa de exploración y análisis de datos.


\chapter{Proceso ETL}

\section{Diseño Proceso ETL}
El diseño de un proceso ETL (Extracción, Transformación y Carga) implica seguir distintos pasos para asegurar que este proceso y el flujo de datos sea eficiente, preciso y cumpla con los requisitos del proyecto, los pasos que se acordaron a seguir son los siguientes:
\begin{itemize}
    \item Requisitos ETL
    \item Identificicación fuente de datos
    \item Diseño modelo de datos objetivo
    \item Planificación de las transformaciones
    \item Selección herramientas
    \item Construcción y prueba proceso ETL
    \item Monitoreo proceso ETL
\end{itemize}


\subsection{Requisitos ETL}
En esta etapa se definen los requisitos del proyecto, las fuentes de datos, los objetivos comerciales y del proceso ETL, las necesidades de análisis y los plazos para realizar el proceso. Estableciendo una base solida para el diseño y buen funcionamiento del proceso ETL.
\begin{itemize}
    \item \textbf{Fuente de datos:} La fuente de datos corresponde a un archivo .CSV que contiene información de la navegación web de los clientes en forma de Web Logs.
    \item \textbf{Objetivos comerciales:} Analizar el comportamiento de los clientes y sus preferencias de uso en un período igual o inferior a 6 meses, para poder predecir navegaciones futuras, y a partir de esto proporcionar atenciones personalizadas.
    \item \textbf{Objetivos proceso ETL:} Realizar las transformaciones necesarias para asegurar que el flujo de datos sea eficiente y preciso, a través de la limpieza de los datos, la normalización, la agregación, el filtrado, el enriquecimiento de datos, así como los cálculos y derivaciones necesarios.
    \item \textbf{Necesidades de análisis:} Realizar un análisis exploratorio de los datos entregados.
\end{itemize}


\subsection{Identificicación fuente de datos}
En esta etapa se determinan las fuentes de datos a ser usadas para el proyecto, incluyendo bases de datos y archivos .CSV y APIs. Esto además comprende la estructura la definición de la estructura, el formato y ubicación de cada fuente de datos dentro del proyecto.
La fuente de datos corresponde a un archivo .CSV que contiene información de la navegación web de los clientes en forma de Web Logs. Los Web Logs registrados vienen con 4 atributos, especificados a continuación
\begin{itemize}
    \item rut cliente: Este atributo representa un identificador único por cliente.
    \item fecha evento: Representa la fecha y hora de la interacción del cliente con el sitio web.
    \item metodo: Este atributo representa cual fue el método al cual el cliente llamo a la hora de interactuar con el sitio web.
    \item canal: Corresponde al canal web con el cual el cliente realizo la interacción con el sitio web.
\end{itemize}

Para almacenar la fuente de datos se utiliza una estructura de carpetas, estas siendo:
\begin{itemize}
    \item Input: Dentro de esta carpeta se encontrara el archivo .CSV tal cual es entregado.
    \item Intermediate: Aquí se almacenará el archivo con la información preprocesada.
    \item Output: Dentro de esta ultima carpeta se almacenara la información ya procesada y lista para ser usada.
\end{itemize}

\subsection{Diseño del modelo de datos objetivo}
En esta etapa, se lleva a cabo el diseño del modelo de datos objetivo. Dado el tipo de proyecto, no será necesario crear un modelo multidimensional u algo similar para su implementación. Esto se debe a que los datos de entrada para los algoritmos predictivos consistirán en datos planos, en forma de Dataframes de pandas.

\subsection{Planificación de las transformaciones}
Dentro de esta etapa, se lleva a cabo la planificación detallada de las transformaciones necesarias para construir una base sólida y consistente que respalde el desarrollo del proyecto. Estas transformaciones comprenden una serie de pasos destinados a limpiar, filtrar, combinar y enriquecer los datos de manera apropiada.

La planificación de las transformaciones desempeña un papel fundamental para asegurar la calidad y la integridad de los datos que se utilizarán en el proyecto. Durante esta fase, se identifican las tareas específicas que deben ejecutarse para lograr los objetivos establecidos, teniendo en cuenta los requisitos del proyecto y las necesidades del negocio.

Algunas de las transformaciones comunes incluyen \cite{etl-toolkit}:

\begin{itemize}
    \item \textbf{Limpieza de datos:} Se llevan a cabo tareas de limpieza para corregir errores, eliminar valores duplicados o inconsistentes y garantizar la coherencia de los datos. En este caso, la limpieza de datos se utilizará para estandarizar el formato y adicionalmente se presentará una propuesta para realizar la transformación y codificación de los campos. Inicialmente, se eliminarán todas las filas que contengan valores nulos. Es importante mencionar que algunos modelos trabajan solamente con datos numéricos por lo que la codificación de campos es opcional según el modelo con el que se quiera trabajar, en este caso algunos modelo de clasificación no requieren dicha codificación por lo que esta no es obligatoria y se abordará de forma detallada en los anexos. %La columna 'fecha' se separará en: año, mes, día, hora, minuto y segundo, y posteriormente se eliminará la columna 'fecha'. Además, se codificarán los campos 'método' y 'canal' utilizando la 'codificación de etiqueta', como se señaló anteriormente. La lógica implementada en esta codificación asignará valores numéricos de forma ascendente a medida que aparezcan en el DataFrame, facilitando así la construcción de un modelo de aprendizaje automático.
    La columna fecha se estandarizará con un formato determinado (UTC) YYYY-MM-DDTHH:MM:SS.sssZ.

    \item \textbf{Filtrado de datos:} Se aplican filtros para seleccionar y extraer los datos relevantes para el proyecto, descartando aquellos que no cumplen con ciertos criterios o condiciones específicas. Se excluyen registros sin métodos asociados o con métodos inconsistentes. Además, se excluyen los registros que contienen el campo de RUT no anonimizado, cumpliendo con la política de privacidad de la empresa. Este enfoque permite centrarse en la información más relevante y útil.
\end{itemize}

Es importante destacar que la planificación de las transformaciones considera el orden y la secuencia adecuada de ejecución, así como la documentación de cada paso y los criterios de validación y verificación para garantizar la calidad de los datos transformados. Por lo tanto, se ha diseñado un plan detallado con las transformaciones que se presentará en la siguiente sección.

\subsection{Selección herramientas}
En esta etapa, se realiza la selección de herramientas de software que se ajusten a las necesidades y requisitos del proyecto para llevar a cabo el proceso ETL de manera eficiente. Se evalúan diferentes opciones disponibles en función de su capacidad, compatibilidad y facilidad de uso, para garantizar una elección adecuada, es por esto que se seleccionaron las siguientes herramientas:
\begin{itemize}
    \item \textbf{Colab}: También conocido como "Colaboratory", permite programar y ejecutar Python en el navegador con las siguientes ventajas \cite{colab}:
    \begin{itemize}
        \item No requiere configuración
        \item Acceso a GPUs sin coste adicional
        \item Permite compartir contenido fácilmente
    \end{itemize}
    Una de las desventajas de usar Colab es que tiene una cantidad de memoria limitada, es decir, si se utiliza el total de memoria disponible no se podrá seguir ejecutando código.
    \item \textbf{Visual Studio Code:} Es un editor de código fuente desarrollado por Microsoft. Es conocido por su enfoque en la simplicidad, la personalización y la eficiencia. Este editor se utilizará en caso de no poder seguir utilizando Colab.
    \item \textbf{Python:} Es un lenguaje de programación interpretado, de alto nivel y de propósito general, conocido por su sintaxis clara y legible. Es utilizado en una amplia gama de aplicaciones, desde desarrollo web hasta ciencia de datos y aprendizaje automático. Python se destaca por su facilidad de aprendizaje y su amplia biblioteca estándar, que ofrece numerosas funcionalidades predefinidas para diversas tareas \cite{python}.
    \begin{itemize}
        \item \textbf{Numpy:} Es una biblioteca fundamental para la computación científica en Python. Proporciona estructuras de datos eficientes y funciones para realizar operaciones numéricas y de manipulación de arrays \cite{numpy}.
        \item \textbf{Pandas:} Es una biblioteca poderosa para el análisis de datos basada en Numpy, que proporciona estructuras de datos flexibles y eficientes, como DataFrames, y un conjunto completo de funciones para la manipulación y transformación de datos \cite{pandas}.
        \item \textbf{Dask:} Es una biblioteca de paralelización flexible que permite escalar el procesamiento de datos en Python. Proporciona estructuras de datos paralelas y operaciones distribuidas que facilitan el procesamiento de grandes volúmenes de datos \cite{dask}.
    \end{itemize}
\end{itemize}


\subsection{Construcción y prueba proceso ETL}
Es en esta etapa en la cual se implementa el diseño del proceso ETL ya definido en puntos anteriores utilizando las herramientas seleccionadas, desarrollando los flujos de extracción, transformación y carga de los datos según lo establecido.
Luego se realizan distintas pruebas para asegurar el correcto funcionamiento del proceso y que se obtengan los resultados esperados.

\subsection{Monitoreo proceso ETL}
Se establece un sistema de monitoreo para poder supervisar el rendimiento del proceso ETL, logrando identificar posibles problemas y garantizar la calidad de los datos. Es en esta etapa donde se hace un mantenimiento del proceso, pudiendo tener actualizaciones de las transformaciones, resolución de problemas y optimizar el proceso.

\chapter{Exploratory Data Analysis (EDA)}

\section{Introducción al EDA}
%Breve descripción del objetivo del análisis exploratorio de datos.
%Explicación de la importancia de comprender los datos antes de construir el modelo de predicción de comportamiento de los clientes.
El análisis exploratorio de datos (EDA, por sus siglas en inglés, Exploratory Data Analysis) es una fase fundamental en la investigación y comprensión de un conjunto de datos. Como su nombre lo indica, el EDA tiene como objetivo explorar y examinar los datos de manera detallada, utilizando resúmenes numéricos y visuales, con el fin de descubrir patrones, tendencias y características no anticipadas. Es considerado uno de los primeros pasos en el proceso de análisis, ya que proporciona una visión general de los datos antes de realizar un análisis más profundo \cite{ruiz2022exploratorio}.

El enfoque principal del EDA radica en el uso de herramientas y técnicas visuales y gráficas para revelar información clave sobre los datos en estudio \cite{parra2002exploratorio}. Estas técnicas incluyen el diagrama de tallo y hoja, el diagrama de caja y bigotes, y el diagrama de dispersión, entre otros. Al aplicar estas técnicas de análisis gráfico, podemos obtener una comprensión más profunda de la distribución y estructura de los datos, así como identificar relaciones entre las variables de interés. Además, el EDA nos brinda la capacidad de detectar posibles errores o puntos extremos, como anomalías, que podrían afectar la calidad de los resultados del análisis.

Los beneficios clave del análisis exploratorio de datos son los siguientes:
\begin{itemize}
    \item \textbf{Conocer la distribución y estructura de los datos:} El EDA nos permite examinar la distribución de las variables y comprender cómo se organizan y dispersan los datos en el conjunto. Esto es fundamental para seleccionar las técnicas adecuadas de análisis estadístico y modelado.
    \item \textbf{Estudiar la relación entre variables:} Mediante el análisis de correlación y la visualización de patrones en los diagramas de dispersión, podemos explorar las relaciones entre las variables y comprender cómo interactúan entre sí. Esto nos brinda información valiosa para identificar posibles dependencias y tendencias en los datos.
    \item \textbf{Encontrar posibles errores y anomalías:} El EDA nos ayuda a identificar valores atípicos, datos faltantes u otros errores en los datos. Estas anomalías pueden tener un impacto significativo en los resultados del análisis, por lo que es importante detectarlas y tratarlas de manera adecuada.
\end{itemize}

\section{Recopilación de datos}
Los datos recopilados de los registros de navegación de los afiliados de AFP Capital constituyen una valiosa fuente de información para comprender el comportamiento y las preferencias de los usuarios en la plataforma web. Estos registros nos permiten analizar cómo interactúan los afiliados con los diferentes canales y métodos disponibles, así como realizar un seguimiento detallado de las fechas y horarios en que se llevan a cabo estas interacciones.

El dataset entregado para este proyecto cuenta con 2,331,023 registros de navegación de usuarios, lo cual proporciona una cantidad significativa de información para su análisis. Antes de utilizar estos datos, se realizó un proceso de anonimización para proteger la privacidad de los usuarios, específicamente modificando el campo del rut para no mostrar el dato original. De esta manera, se garantiza que los registros sean tratados de forma confidencial y segura.

Los cuatro campos principales que conforman el conjunto de datos son el rut, la fecha del evento, el método y el canal. El rut, que ha sido modificado, actúa como un identificador único para cada usuario y permite realizar análisis individuales sin revelar su identidad. La fecha del evento registra el momento exacto en que se llevó a cabo cada navegación, lo cual es crucial para identificar patrones y tendencias a lo largo del tiempo. El campo del método describe la interacción específica realizada por el usuario en el canal correspondiente, proporcionando información detallada sobre las acciones que realizan. Por último, el campo del canal indica el sitio o ambiente particular en el cual tuvo lugar cada interacción, lo que puede ser útil para comprender las preferencias de los usuarios en relación con los diferentes entornos disponibles.

Con respecto al preprocesamiento de datos realizado hasta la fecha, se ha seguido el proceso ETL (Extracción, Transformación y Carga) que se describe en detalle en el capítulo anterior. Este proceso implica extraer los datos de las fuentes de origen, transformarlos en un formato adecuado y cargarlos en un sistema de almacenamiento para su posterior análisis. Se utilizaron diversas herramientas especializadas para llevar a cabo estas tareas, asegurando la calidad y coherencia de los datos procesados.

Es importante destacar que, si bien los datos recopilados ofrecen una valiosa perspectiva sobre el comportamiento de los usuarios en la plataforma web, es necesario tener en cuenta que existe un sesgo en la muestra de datos. En particular, los registros de navegación corresponden principalmente a afiliados con rentas altas. Esto implica que los usuarios con ingresos más altos, aquellos que cotizan por un valor elevado o el valor máximo, están sobrerrepresentados en la muestra. Por lo tanto, al interpretar y generalizar los resultados obtenidos, es fundamental tener en cuenta esta limitación y considerar posibles variaciones en el comportamiento de otros segmentos de usuarios.

\section{Estadísticas descriptivas}
Resumen de las principales características de las variables relevantes en los registros de logs de navegación. Análisis de la distribución de los datos, incluyendo medidas de centralidad y dispersión. Usando la librería Pandas de Python, podemos obtener información importante sobre los conjuntos de datos. Estos valores nos proporcionarán un mayor entendimiento de la composición y estructura de los registros de navegación.

Podemos obtener información respecto a la cantidad de valores faltantes en el dataframe que se está utilizando, los cuales suman 36,853 en total. De estos datos faltantes, 36,853 corresponden principalmente al campo 'canal', con solo un valor faltante en el campo 'método'. Como se vio en el capítulo anterior, estos valores fueron eliminados del dataframe.

Haciendo uso de la misma librería, podemos determinar los tipos de datos que contiene cada campo:

\begin{itemize}
    \item \textbf{rut:} Contiene datos de tipo string.
    \item \textbf{fecha:} Contiene datos de tipo date.
    \item \textbf{metodo:} Contiene datos de tipo string.
    \item \textbf{canal:} Contiene datos de tipo string.
\end{itemize}

Además se puede conocer la cantidad de valores únicos que posee cada una de las columnas del dataframe (luego del proceso ETL):

\begin{itemize}
    \item \textbf{rut cliente:} Contiene 35,524 valores únicos.
    \item \textbf{fecha:} Contiene 21,148 valores únicos.
    \item \textbf{metodo:} Contiene 85 valores únicos.
    \item \textbf{canal:} Contiene 2 valores únicos.
\end{itemize}

Ahora veremos la frecuencia con la que aparecen los valores más frecuentes en cada campo:
\begin{itemize}
    \item Para rut el valor más común es 'MTU0NjM5NzIx' y tiene una frecuencia de 320,689.
    \item Para método el valor más común es 'getAccountWithdrawal()' y tiene una frecuencia de 259,733.
    \item Para el canal el más común es 'PWA' con una frecuencia de 1,029,538.
    \item Para el campo fecha es distinto ya que cada registro posee valores distintos, esto se debe al detalle del valor guardado ya que este llega a considerar las milesimas de segundos.
\end{itemize}

Ahora veremos un gráfico que representa la distribución de los métodos en el conjunto de datos. En esta ocasión, se presentarán únicamente los diez métodos que tienen la mayor frecuencia, dado que la cantidad de métodos registrados es demasiado grande para ser mostrada en un solo gráfico.

\begin{figure}[H]
    \begin{minipage}[t]{0.9\textwidth}
        \caption{Distribución de métodos en el dataset.}
        \label{frecuencia-metodos}        
    \end{minipage}

    \vspace{10pt}

    \begin{minipage}[b]{1.0\textwidth}
        \centering
        \includegraphics[width=\textwidth]{img/frecuencia-metodo.png}        
    \end{minipage}

    \begin{minipage}[t]{0.9\textwidth}
        Fuente: Elaboración propia.
    \end{minipage}
\end{figure}

\sloppy
La distribución muestra que las actividades relacionadas con cuentas (getAccountWithdrawal(), getAccounts(), y getAccountsConditions()) son las más frecuentes. Esto sugiere que la mayoría de los usuarios están interesados en la gestión y consulta de sus cuentas. Por otro lado, las actividades como login() y loginApplication() también tienen una alta frecuencia, indicando que el proceso de inicio de sesión es común, lo cual es esperado en una plataforma segura. De acá podemos podemos concluir que las actividades más populares pueden ser un área clave para la mejora continua y la optimización de la experiencia del usuario. Las actividades menos frecuentes no deben ser ignoradas; se debe evaluar su importancia y cómo pueden ser promocionadas o mejoradas.

Para analizar las fechas, se decidió realizar una descripción de estas por día y hora. Año y mes quedan fuera del análisis ya que todos los datos pertenecen al mismo año y mes. Respecto a los días, en el set de datos hay un total de ocho días, y están presentes las veinticuatro horas del día. Estas tienen una frecuencia de:

\subsubsection{Distribución de días:}
\begin{itemize}
    \item Se cuenta con un rango de 8 días consecutivos, comenzando en el día 14 y finalizando en el día 20. Cada día está asociado con un número que refleja una cantidad particular, con el día 14 teniendo la mayor cantidad de 4689, y el día 20 la menor con 138.
    % \item 14: 4689
    % \item 13: 4598
    % \item 15: 4026
    % \item 16: 2991
    % \item 17: 2619
    % \item 18: 1877
    % \item 19: 211
    % \item 20: 138
\end{itemize}

\subsubsection{Distribución de horas:}
\begin{itemize}
    \item Se observa una distribución de métricas a lo largo de las 24 horas del día. Las horas con las mayores cantidades se presentan al inicio de la tarde, con la hora 14 liderando con 1382 unidades. La actividad disminuye progresivamente hacia las horas nocturnas, siendo la hora 22 la última con una cantidad significativa de 1055 unidades antes de una reducción notable durante las horas de madrugada. Las horas menos activas son las primeras horas de la mañana, con la hora 5 registrando la menor cantidad de 438 unidades.
    % \item 14: 1382
    % \item 15: 1263
    % \item 13: 1240
    % \item 16: 1123
    % \item 22: 1055
    % \item 19: 1042
    % \item 18: 1037
    % \item 21: 1022
    % \item 3: 1009
    % \item 12: 999
    % \item 23: 992
    % \item 20: 985
    % \item 17: 921
    % \item 00: 890
    % \item 1: 859
    % \item 2: 800
    % \item 4: 787
    % \item 11: 726
    % \item 10: 605
    % \item 9: 540
    % \item 8: 514
    % \item 6: 463
    % \item 7: 457
    % \item 5: 438
\end{itemize}

Al analizar la distribución de los días, se observa que los días 14 y 13 presentan la mayor frecuencia en los registros, con 4689 y 4598 eventos, respectivamente. A medida que avanzan los días, se nota una disminución significativa en la cantidad de registros, llegando a solo 138 eventos el día 20.

En cuanto a la distribución de las horas, se puede observar que las horas 14, 15 y 13 tienen las frecuencias más altas, con 1382, 1263 y 1240 eventos, respectivamente. Las horas entre las 16 y las 21 también muestran una frecuencia considerablemente alta, mientras que las horas más tempranas (de 00 a 04) y las horas de la madrugada (después de las 04) presentan menor actividad, siendo la hora 05 la menos frecuente con 438 eventos.

Este análisis detallado por días y horas proporciona una visión más completa sobre la distribución temporal de los eventos registrados en el conjunto de datos, lo que puede ser útil para comprender los patrones de comportamiento de los usuarios en la plataforma.

% \subsection{Visualizaión de datos}
% Utilización de gráficos y visualizaciones para explorar los datos.
Representación visual de las características relevantes de los clientes y su comportamiento en el canal web.



%Análisis de patrones, tendencias o relaciones identificadas a través de las visualizaciones.

%\subsection{Análisis de correlación}
%%Exploración de la relación entre las variables relevantes en los logs de navegación.
%Cálculo de coeficientes de correlación u otras medidas para evaluar la fuerza y dirección de las relaciones.
%Interpretación de los resultados obtenidos y discusión sobre cómo pueden influir en el modelo de predicción.

%\subsection{Análisis de variables importantes}
%%Identificación de las variables más relevantes o influyentes en el comportamiento de los clientes.
%Uso de técnicas estadísticas o algoritmos de selección de características para determinar la importancia relativa de las variables.
%Discusión de los hallazgos y cómo se utilizarán en el modelo de predicción.

%\subsection{Resultados}
%%Resumen de los principales hallazgos del análisis exploratorio de datos.
%Discusión de las implicaciones de estos hallazgos en el proyecto de %empresa y el modelo de predicción de comportamiento de los clientes.
%Posibles limitaciones del análisis exploratorio y propuestas de futuras investigaciones.

\begin{doublespace}
  \bibliographystyle{apacite}
  \bibliography{Bibliografia}
\end{doublespace}
\end{document}